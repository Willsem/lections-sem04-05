\documentclass[a4paper, 14pt]{report}

\usepackage{cmap}
\usepackage[T2A]{fontenc}
\usepackage[utf8]{inputenc}
\usepackage[english,russian]{babel}
\usepackage[left=30mm, top=20mm, right=20mm, bottom=20mm, nohead, nofoot]{geometry}
\usepackage{indentfirst}

\usepackage{amsmath}
\usepackage{MnSymbol}
\usepackage{wasysym}

\usepackage{pgfplots}

\author{Ульянов Михаил Васильевич}
\title{Анализ алгоритмов}
\date{2019}

\begin{document}
    \maketitle

    \tableofcontents
    \clearpage

    \chapter{Исторический очерк}

    \begin{enumerate}
        \item \textbf{1900} \newline
            Д. Гильберт - 23 проблемы
            1931 - К. Гедель доказал теорему о неполноте
        \item \textbf{1936} \newline
            А.Тьюринг, Э.Л.Пост - Теория алгоритмов (начало)
        \begin{enumerate}
            \item[-] формализация понятия
            \item[-] общие свойства
            \item[-] обнаружение алгоритмически неразрешимых задач
        \end{enumerate}
        \item \textbf{1960е} \newline
            Теория сложности вычислений NPC \newline
            $O(n^2)\ O(n \cdot \ln(n))$

            \begin{tikzpicture}
                \begin{axis}[
                    xlabel = {$n$},
                    minor tick num = 10,
                    domain=0:10,
                    legend pos = north west,
                    grid = major,
                    line width = 1
                ]
                    \legend{
                        $n^2$,
                        $n \cdot \ln(n)$
                    };

                    \addplot[blue] {x^2};
                    \addplot[orange] {x*ln(x)};
                \end{axis}
            \end{tikzpicture}

        \item \textbf{Начало 1970х} \newline
            Практический анализ алгоритмов Д.Э. Кнут
    \end{enumerate}

    \chapter{Схема выбора алгоритмического обеспечения}

    Нет:

    \begin{enumerate}
        \item[A.] Новый (метод разработки)
        \item[B.] Комбинированные элементы ($A_1 + A_2 + A_3$)
    \end{enumerate}

    $$ Q(q_1,...,q_m) = \sum \alpha_i q_i \to R^1 \text{ - комплексные оценки}$$

    $$
    \begin{cases}
        a_{11} x_1 + a_{12} x_2 = b_1 \\
        a_{21} x_1 + a_{22} x_2 = b_2
    \end{cases}
    $$

    $$ a + ib = (c + id) = ^{det} (ac - bd) + i (bc + ad) $$
\end{document}
