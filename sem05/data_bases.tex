\documentclass[a4paper, 14pt]{report}

\usepackage{cmap}
\usepackage[T2A]{fontenc}
\usepackage[utf8]{inputenc}
\usepackage[english,russian]{babel}
\usepackage[left=30mm, top=20mm, right=20mm, bottom=20mm, nohead, nofoot]{geometry}
\usepackage{indentfirst}

\usepackage{amsmath}
\usepackage{MnSymbol}
\usepackage{wasysym}

\usepackage{pgfplots}

\usepackage{tikz}
\usetikzlibrary{graphs}

\usepackage[most]{tcolorbox}
\newtcolorbox{lbox}[2][] {
    enhanced,
    %fonttitle=\ttfamily,
    %fontupper=\ttfamily,
    sharp corners,
    colback=white,
    colbacktitle=white,
    coltitle=black,
    boxed title style={colframe=white},
    attach boxed title to top center={yshift=-3mm},
    title=#2,#1
}

\author{Гаврилова Юлия Михайловна}
\title{Базы данных}
\date{2019}

\begin{document}
    \maketitle

    \tableofcontents
    \clearpage

    \chapter{Введение}

    \begin{center}
    \begin{tabular}{|c|c|}
        \hline
        \multicolumn{2}{|c|}{Способы организации} \\
        \hline
        \textbf{OLAP} (online analytic processor) & \textbf{OLTP} (jnline transaction processor) \\
        \hline
        Время отклика & Быстрая вставка \\
        \hline
        3NF & 1NF \\
        \hline
        Нормальная форма & Для сбора статистики \\
        \hline
    \end{tabular}
    \end{center}

    \begin{tikzpicture}
        \graph[nodes={align=center,rectangle,draw=black}, grow down sep, branch right sep]
        {
            SQL -> "Reluationnaya model" ->
            {
                "Teoria mnozestv",
                "Teoria predikatov"
            }
        };
    \end{tikzpicture}

    \hfill

    \begin{tikzpicture}
        \graph[nodes={align=center,rectangle,draw=black}, grow down sep, branch right sep]
        {
            SYBD -> BD
        };
    \end{tikzpicture}

    \section{Реляционная модель}

    \begin{enumerate}
        \item Стректурная часть: как построена модель
        \item Целостная часть: какие ограничения, как должны быть организованы данные
        \item Манипуляционная: обработка данных
    \end{enumerate}

    \subsection{Структурная часть}

    \begin{itemize}
        \item Тип int, char
        \item домен - надстройка над типом, набор ограничений/правил (положительные четные для int), можно объявить над типом или над доменом
        \item атрибут - упорядоченная пара (<имя, тип или домен>)
        \item заголовок (схема) отношения - множество всех пар атрибутов \{<имя атрибута$_1$, значение$_1$>,$\dots$, <имя атрибута$_N$, значение$_N$>\}

            \{<$a_1$, int>,<$a_2$, float>,<$a_3$, char>,<$a_4$, varchar>\}
        \item кортеж над схемой

            \{<$a_1$, 1>,<$a_2$, 1.4>,<$a_3$, 'a'>,<$a_4$, 'aaa'>\}
        \item отношение

            \begin{tabular}{|c|c|c|c|}
                \hline
                $a_1$ & $a_2$ & $a_3$ & $a_4$ \\
                \hline
                1 & 1.4 & 'a' & 'aaa' \\
                \hline
            \end{tabular}
    \end{itemize}


    \paragraph{ER-модель}

    \begin{itemize}
        \item отношение/сущность

            \begin{tikzpicture}
                \graph[nodes={align=center,rectangle,draw=black}, grow down sep, branch right sep]
                {
                    students -> "second name",
                    students -> "group",
                    students -> tails,
                    students -> number
                };
            \end{tikzpicture}
    \end{itemize}

    Здесь студент сущность сильная. Если студент зависит, то студент - слабая сущность

    \begin{itemize}
        \item связь 1 - 1 (Студент $\to$ зачетка)
        \item связь 1 ко многим (Студенты $\to$ группа)
        \item многие ко многим (Студенты $\to$ курс)
    \end{itemize}

    \begin{lbox}{\textbf{Лабораторная работа 1}}
        \begin{itemize}
            \item Подобрать предметную область на весь семестр
            \item ER модель (не менее 3ч самостоятельных сущностей)
            \item Создать свою БД (не менее 1000 записей на таблицу)
        \end{itemize}
        \textbf{Защита:}
        \begin{itemize}
            \item Добавить связь/атрибут
            \item Создать ссылку
        \end{itemize}
    \end{lbox}

    \subsection{Целостная часть}

    \begin{itemize}
        \item целостность сущностей/отношений
        \item целостность ссылок

            \begin{tabular}{|c|c|c|}
                \hline
                id & ФИО & Age \\
                \hline
                1 & Иванов & 10 \\
                \hline
                2 & Петров & 15 \\
                \hline
                3 & Иванов & 45 \\
                \hline
            \end{tabular}
    \end{itemize}

    Потенциальный ключ:

    \begin{itemize}
        \item однозначная идентификация записи
        \item никаких подмножеств не должно быть под ключом
    \end{itemize}

    \begin{center}
        \begin{tabular}{|c|c|c|}
            \hline
            id & ФИО & id группы \\
            \hline
            1 & Петров & 1 \\
            \hline
            & & \\
            \hline
        \end{tabular}

        $\downarrow$ Внешняя ссылка

        \begin{tabular}{|c|c|}
            \hline
            id & Название \\
            \hline
            1 & ИУ7-53 \\
            \hline
            & \\
            \hline
        \end{tabular}
    \end{center}

    Ссылочная целостность - нельзя ссылаться на несуществующий объект

    \subsection{Манипуляционная часть}

    \begin{itemize}
        \item Реляционная алгебра
        \item Реляционные исчисления
    \end{itemize}

\end{document}
