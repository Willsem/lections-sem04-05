\documentclass[a4paper, 14pt]{report}

\usepackage{cmap}
\usepackage[T2A]{fontenc}
\usepackage[utf8]{inputenc}
\usepackage[english,russian]{babel}
\usepackage[left=30mm, top=20mm, right=20mm, bottom=20mm, nohead, nofoot]{geometry}
\usepackage{indentfirst}

\usepackage{amsmath}
\usepackage{MnSymbol}
\usepackage{wasysym}

\usepackage{pgfplots}

\usepackage{tikz}
\usetikzlibrary{graphs}

\usepackage[most]{tcolorbox}
\newtcolorbox{lbox}[2][] {
    enhanced,
    %fonttitle=\ttfamily,
    %fontupper=\ttfamily,
    sharp corners,
    colback=white,
    colbacktitle=white,
    coltitle=black,
    boxed title style={colframe=white},
    attach boxed title to top center={yshift=-3mm},
    title=#2,#1
}

\author{Гаврилова Юлия Михайловна}
\title{Базы данных}
\date{2019}

\begin{document}
    \maketitle

    \tableofcontents
    \clearpage

    \chapter{Введение}

    \begin{center}
    \begin{tabular}{|c|c|}
        \hline
        \multicolumn{2}{|c|}{Способы организации} \\
        \hline
        \textbf{OLAP} (online analytic processor) & \textbf{OLTP} (jnline transaction processor) \\
        \hline
        Время отклика & Быстрая вставка \\
        \hline
        3NF & 1NF \\
        \hline
        Нормальная форма & Для сбора статистики \\
        \hline
    \end{tabular}
    \end{center}

    \begin{tikzpicture}
        \graph[nodes={align=center,rectangle,draw=black}, grow down sep, branch right sep]
        {
            SQL -> "Reluationnaya model" ->
            {
                "Teoria mnozestv",
                "Teoria predikatov"
            }
        };
    \end{tikzpicture}

    \hfill

    \begin{tikzpicture}
        \graph[nodes={align=center,rectangle,draw=black}, grow down sep, branch right sep]
        {
            SYBD -> BD
        };
    \end{tikzpicture}

    \section{Реляционная модель}

    \begin{enumerate}
        \item Стректурная часть: как построена модель
        \item Целостная часть: какие ограничения, как должны быть организованы данные
        \item Манипуляционная: обработка данных
    \end{enumerate}

    \subsection{Структурная часть}

    \begin{itemize}
        \item Тип int, char
        \item домен - надстройка над типом, набор ограничений/правил (положительные четные для int), можно объявить над типом или над доменом
        \item атрибут - упорядоченная пара (<имя, тип или домен>)
        \item заголовок (схема) отношения - множество всех пар атрибутов \{<имя атрибута$_1$, значение$_1$>,$\dots$, <имя атрибута$_N$, значение$_N$>\}

            \{<$a_1$, int>,<$a_2$, float>,<$a_3$, char>,<$a_4$, varchar>\}
        \item кортеж над схемой

            \{<$a_1$, 1>,<$a_2$, 1.4>,<$a_3$, 'a'>,<$a_4$, 'aaa'>\}
        \item отношение

            \begin{tabular}{|c|c|c|c|}
                \hline
                $a_1$ & $a_2$ & $a_3$ & $a_4$ \\
                \hline
                1 & 1.4 & 'a' & 'aaa' \\
                \hline
            \end{tabular}
    \end{itemize}


    \paragraph{ER-модель}

    \begin{itemize}
        \item отношение/сущность

            \begin{tikzpicture}
                \graph[nodes={align=center,rectangle,draw=black}, grow down sep, branch right sep]
                {
                    students -> "second name",
                    students -> "group",
                    students -> tails,
                    students -> number
                };
            \end{tikzpicture}
    \end{itemize}

    Здесь студент сущность сильная. Если студент зависит, то студент - слабая сущность

    \begin{itemize}
        \item связь 1 - 1 (Студент $\to$ зачетка)
        \item связь 1 ко многим (Студенты $\to$ группа)
        \item многие ко многим (Студенты $\to$ курс)
    \end{itemize}

    \begin{lbox}{\textbf{Лабораторная работа 1}}
        \begin{itemize}
            \item Подобрать предметную область на весь семестр
            \item ER модель (не менее 3ч самостоятельных сущностей)
            \item Создать свою БД (не менее 1000 записей на таблицу)
        \end{itemize}
        \textbf{Защита:}
        \begin{itemize}
            \item Добавить связь/атрибут
            \item Создать ссылку
        \end{itemize}
    \end{lbox}

    \subsection{Целостная часть}

    \begin{itemize}
        \item целостность сущностей/отношений
        \item целостность ссылок

            \begin{tabular}{|c|c|c|}
                \hline
                id & ФИО & Age \\
                \hline
                1 & Иванов & 10 \\
                \hline
                2 & Петров & 15 \\
                \hline
                3 & Иванов & 45 \\
                \hline
            \end{tabular}
    \end{itemize}

    Потенциальный ключ:

    \begin{itemize}
        \item однозначная идентификация записи
        \item никаких подмножеств не должно быть под ключом
    \end{itemize}

    \begin{center}
        \begin{tabular}{|c|c|c|}
            \hline
            id & ФИО & id группы \\
            \hline
            1 & Петров & 1 \\
            \hline
            & & \\
            \hline
        \end{tabular}

        $\downarrow$ Внешняя ссылка

        \begin{tabular}{|c|c|}
            \hline
            id & Название \\
            \hline
            1 & ИУ7-53 \\
            \hline
            & \\
            \hline
        \end{tabular}
    \end{center}

    Ссылочная целостность - нельзя ссылаться на несуществующий объект

    \subsection{Манипуляционная часть}

    \begin{itemize}
        \item Реляционная алгебра
        \item Реляционные исчисления
    \end{itemize}

    \section{Реляционная алгебра}

    \begin{tabular}{|c|c|}
        \hline
        id & name \\
        \hline
        1 & a \\
        \hline
        2 & b \\
        \hline
    \end{tabular}

    \hfill

    \begin{tabular}{|c|c|}
        \hline
        id & name \\
        \hline
        2 & b \\
        \hline
        3 & c \\
        \hline
    \end{tabular}

    \begin{enumerate}
        \item Традиционные - работа с множеством

            \begin{itemize}
                \item Объединение (UNION)

                    \begin{tabular}{|c|c|}
                        \hline
                        id & name \\
                        \hline
                        1 & a \\
                        \hline
                        2 & b \\
                        \hline
                        3 & c \\
                        \hline
                    \end{tabular}

                \item Пересечение (INTERSECT)

                    \begin{tabular}{|c|c|}
                        \hline
                        id & name \\
                        \hline
                        2 & b \\
                        \hline
                    \end{tabular}

                \item Вычитание (MINUS)

                    \begin{tabular}{|c|c|}
                        \hline
                        id & name \\
                        \hline
                        1 & a \\
                        \hline 
                    \end{tabular}

                    \begin{tabular}{|c|c|}
                        \hline
                        id & name \\
                        \hline
                        3 & c \\
                        \hline 
                    \end{tabular}

                \item Декартово произведение (TIMES) - все возможные комбинации атрибутов
            \end{itemize}

        \item Специальные

            \begin{itemize}
                \item Соединение (JOIN)

                    \begin{tabular}{|c|c|c|}
                        \hline
                        id & name1 & name2\\
                        \hline
                        2 & b & b \\
                        \hline
                    \end{tabular}

                \item Ограничение (WHERE)
                \item Проекция (PROJECT)
                \item Деление (DIVIDE BY)
            \end{itemize}
    \end{enumerate}

    Реляционное выражение = унарное выражение (бинарное выражение)

    \paragraph{Унарные выражения}

    \begin{itemize}
        \item Проекция

            терм | терм[список атрибутов]

        \item Ограничение

            терм WHERE логическое\_выражение

        \item Переименование

            терм RENAME old\_name TO new\_name

    \end{itemize}

    терм - имя\_отношения | (реляционное\_выражение)

    \paragraph{Бинарные выражения}

    \begin{itemize}
        \item Объединение
        \item Пересечение
        \item Вычитание
        \item Декартово произведение
        \item Соединение
    \end{itemize}

    бинарные операции = проекция бинарная\_операция реляцонное\_выажение

    S JOIN P[P..,S..]

    \begin{tabular}{|c|}
        \hline
        Поставщик \\
        \hline
        S \\
        \hline
    \end{tabular}

    $ \downarrow $ Многие ко многим SP

    \begin{tabular}{|c|}
        \hline
        Детали \\
        \hline
        P \\
        \hline
    \end{tabular}

    \hfill

    $S(Sno : integer, Sname : string, Status : integer, City : string)$

    $P(Pno : integer, Pname : string, Color : string, Weight : real, City : string)$

    $SP(Sno : integer, Pno : integer, Quantity : integer)$

    \paragraph{S}

    \hfill

    \begin{tabular}{|c|c|c|c|}
        \hline
        Sno & Sname & Status & City \\
        \hline
        1 & Алмаз & 20 & Смоленск \\
        2 & Дельта & 10 & Владимир \\
        3 & Орион & 30 & Смоленск \\
        \hline
    \end{tabular}

    \paragraph{P}

    \hfill

    \begin{tabular}{|c|c|c|c|c|}
        \hline
        Pno & Pname & Color & Weight & City \\
        \hline
        1 & Гайка & К & 12.0 & Смоленск \\
        2 & Болт & С & 17.1 & Рязань \\
        3 & Винт & З & 15.47 & Владимир \\
        4 & Винт & К & 18 & Москва \\
        5 & Шайба & З & 25 & Смоленск \\
        \hline
    \end{tabular}

    \paragraph{SP}

    \hfill

    \begin{tabular}{|c|c|c|}
        \hline
        Sno & Pno & Quantity \\
        \hline
        1 & 1 & 25 \\
        1 & 2 & 14 \\
        2 & 4 & 2 \\
        \hline
    \end{tabular}

    \begin{enumerate}
        \item Имена всех поставщиков детали под номером 2

            $$
            \underbrace{((\underbrace{\text{SP join S})}_\text{рел. выр.} \text{where} \underbrace{\text{Pno = 2}}_{\text{лог. выр.}})}_\text{реляционное выражение}[\text{Sname}]
            $$

            select Sname

            from SP inner join S on SP.Sno = S.Sno

            where SP.Pno = 2

        \item Вывести все имена поставщиков, которые поставляюк как минимум одну красную деталь

            $$
            (((\text{P where } \text{Color = 'К'}) \text{join SP}) \text{join S})[\text{Sname}]
            $$

        \item Получить имена поставщиков, которые поставляют все детали

            $A(X_1, \dots, X_n, Y_1, \dots, Y_n)$

            $B(Y_1, \dots, Y_n)$

            $A \text{ divide by } B = (X_1, \dots, X_n)$

            \begin{tabular}{|c|c|}
                \hline
                Sno & Pno \\
                \hline
                1 & 1 \\
                1 & 2 \\
                1 & 3 \\
                2 & 2 \\
                2 & 3 \\
                3 & 1 \\
                \hline
            \end{tabular}

            $P[Pno]$

            $SP \text{ divide by } P[Pno]$

            $((SP \text{ divide by } P[Pno]) \text{ join } S)[Sname]$

        \item Все поставщики, которые поставляют только красные детали

            $(Sp \text{divide by }(P \text{ where } \text{Color = 'К'})[Pno])[Sname]$

        \item Переименовать города из первой таблицы во вторые

            $(S \text{ rename } Sno \text{ to } firstName)[firstName, City] \text{ join }$
            
            $(S \text{ rename } Sno \text{ to } secondName)[secondName, City]) \text{ where } secondName > firstName \text{ join } S$

            \begin{tabular}{|c|c|}
                \hline
                firstName & C \\
                \hline
                1 & С \\
                2 & В \\
                3 & С \\
                \hline
            \end{tabular}

            \begin{tabular}{|c|c|}
                \hline
                secondName & C \\
                \hline
                1 & С \\
                2 & В \\
                3 & С \\
                \hline
            \end{tabular}

            \begin{tabular}{|c|c|c|}
                \hline
                firstName & secondName & C \\
                \hline
                1 & 1 & С \\
                1 & 3 & С \\
                2 & 2 & В \\
                3 & 1 & С \\
                3 & 3 & С \\
                \hline
            \end{tabular}

        \item Поставщики, которые не поставляют деталь номер 2

            $((S[Sno] \text{ minus } (SP \text{ where } Pno=2)[Sno]) \text{ join } S)[Sname]$

    \end{enumerate}
\end{document}
