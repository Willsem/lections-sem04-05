\documentclass[a4paper, 14pt]{report}

\usepackage{cmap}
\usepackage[T2A]{fontenc}
\usepackage[utf8]{inputenc}
\usepackage[english,russian]{babel}
\usepackage[left=30mm, top=20mm, right=20mm, bottom=20mm, nohead, nofoot]{geometry}
\usepackage{indentfirst}

\usepackage{amsmath}
\usepackage{MnSymbol}
\usepackage{wasysym}

\usepackage{pgfplots}

\usepackage{tikz}
\usetikzlibrary{graphs}

\usepackage[most]{tcolorbox}
\newtcolorbox{lbox}[2][] {
    enhanced,
    sharp corners,
    colback=white,
    colbacktitle=white,
    coltitle=black,
    boxed title style={colframe=white},
    attach boxed title to top center={yshift=-3mm},
    title=#2,#1
}

\author{Рязанова Наталья Юрьевна}
\title{Операционные системы}
\date{2019}

\begin{document}
    \maketitle

    \tableofcontents
    \clearpage

    \chapter{История}

    Деления на поколения можно считать очень условными.

    \textbf{Компьютер} - программно управляемое устройство.

    Современным ПК часть времени управляет ОС, часть времени - приложения.

    В 40-х годах потребность в больших, быстрых, точных вычислениях увеличилась из-за гонки вооружений.

    \begin{itemize}

        \item В 1944 году в США была создана MARK I - первая вычислительная машина на электромагнитных реле.

        \item В 1946 году - первая электронно-цифровая машина на электро магнитных лампах - UNIVAK.

        \item 1945-46 гг. - создание первого поколения ЭВМ. Длился до 1955 года. Особенности: использовались электронные лампы, ЗУ на линиях задержки, ЗУ вращающегося типа, концепция хранения программ. Для ввода/вывода - перфокарты, печатающее устройство.
    \end{itemize}

    Первые серийные машины - MARK I, UNIVAK I, LEO I.

    Все программы выполнялись в \textbf{абсолютных адресах} - адресах байта/слова (2 байта) и т.д. физической памяти. В эти же годы разработали ENIAC. В группу разработки в 1944 г. вошел Джон Фон Нейман. В 1945 г. он опубликовал доклад, в котором определены основные принципы построения компьютерной машины, которую называют компьютером. В 1946 г. - статья "Предварительная конструкция ЭВУ", в ней была описана формальная огранизация работы машины.

    \textbf{Принцип хранимой программы} - данные и команды хранятся в одной памяти. Для того, чтобы к ним отбращаться, они хранятся в опредеенных адресах. Требует это счетчик команд, он хранит адрес следующей команды.

    С середины 50-х годов - отсчет второго поколения ЭВМ. Появились диоды и триоды (транзисторы) и ОЗУ на магнитных сердечниках.

    Для серийного производства машины нужна техническая документация.

    \begin{tikzpicture}
        \graph[nodes={rectangle,draw=black}, grow right sep=1em]{
            IBM 1401 -> IBM 7094 -> 1401
        };
    \end{tikzpicture}

    Для автоматического перескока с 1 задания на другое понадобилось специальное ПО, которое стали называть ОС. Была создана серия оберточных и отладочных программ, которые помогали программису. Для управления заданиями был разработан специальный язык - язык управления заданиями.

    С начала 60-х годов - 3 поколение ЭВМ. Появились микросхемы. К 3 поколению относится полноценное появление \textbf{архитектуры ЭВМ}. Появилось в связи с идеей мультипрограммными обработками и реализацией распаралеливания фукнций. Для ускорения переключения между заданиями - загрузка нескольких заданий в ОЗУ.

    Большое число программ разделяет одно адресное пространство физической памяти.

    \textbf{Процесс} - программа в стадии выполнения.

    Перед ОС появились новые задачи, появился менеджер памяти для выделения/распределения/очищения. Нужно было обеспечить защиту защиту памяти выделенной для программы.

    В 3 поколении появилась начальная архитектура. \textbf{Канал} - программно управляемое устройство, в задачи когорого входило управление внешними устройствами.

    IBM 360 - машина третьего поколения.

    \hfill

    \begin{tikzpicture}
        \graph[nodes={align=center,rectangle,draw=black}, grow down sep, branch right sep]
        {
            Proc -> Memory ->
            {
                "Multiplex channel" -> YVV1,
                "Spector channel" -> YVV2,
                "BLOK-multiplex channel" -> YVV3
            },
            "Multiplex channel" -> Memory,
            "Spector channel" -> Memory,
            "BLOK-multiplex channel" -> Memory,
            "Multiplex channel" -> Proc,
            "Spector channel" -> Proc,
            "BLOK-multiplex channel" -> Proc,
            YVV1 -> "Multiplex channel",
            YVV2 -> "Spector channel",
            YVV3 -> "BLOK-multiplex channel"
        };
    \end{tikzpicture}
\end{document}
