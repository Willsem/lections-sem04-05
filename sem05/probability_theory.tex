\documentclass[a4paper, 14pt]{report}

\usepackage{cmap}
\usepackage[T2A]{fontenc}
\usepackage[utf8]{inputenc}
\usepackage[english,russian]{babel}
\usepackage[left=30mm, top=20mm, right=20mm, bottom=20mm, nohead, nofoot]{geometry}
\usepackage{indentfirst}

\usepackage{amsmath}
\usepackage{MnSymbol}
\usepackage{wasysym}

\usepackage{amsthm}
\usepackage{mdframed}

\usepackage{pgfplots}
\pgfplotsset{compat=1.9}

\newmdtheoremenv{theorem}{Теорема}[chapter]
\newtheorem{lemma}{Лемма}[chapter]
\newtheorem{defenition}{Определение}[chapter]
\newtheorem{note}{Замечание}[chapter]
\newtheorem{cont}{Следствие}[chapter]

\author{Власов Павел Александрович}
\title{Теория вероятностей}
\date{2019}

\begin{document}
\maketitle

\tableofcontents
\clearpage

\chapter{Двойной интеграл}

\section{Площадь плоской фигуры}

Пусть $D$ - фигура на плоскости.

Как ввести понятие площади фигуры $D$?

Если $D$ является треугольником (или прямоугольником) понятие площади очевидно.

Если $D$ является многоугольником, то ее можно разбить на треугольники, а площадь области $D$ определить как сумму состовляющих ее треугольников.

Что делать если $D$ - произвольная фигура

\begin{enumerate}

    \item[а)] Рассмотрим множество многоугольников $M$, каждый из которых целиком содержится в $D$.

    Обозначим $S_* = \text{sup}\ S(m)$, где $m$ - многоугольники, $S(m)$ - площадь многоугольника

    \item[б)] Рассмотрим множество многоугольников $M$, каждый из которых содержит в себе $D$.

    Обозначим $S^* = \text{suf}\ S(M)$

\end{enumerate}

\begin{defenition}
    Область $D$ на плоскости называется \textbf{квадрируемой}, если $\exists$ конечные значения $S_*$, $S^*$ причем $S_* = S^*$. При этом число $S = S_* = S^*$ называется \textbf{площадью области $D$}
\end{defenition}

\hfill

\begin{defenition}
    Говорят, что множество $D$ точек плоскости имеет площадь \textbf{нуль}, если $D$ можно целиком заключить в многоугольник, сколь угодной площади, т.е. $\forall \varepsilon > 0\ \exists$ многоугольник $M$ площади $\varepsilon$ такой, что $D \le M$
\end{defenition}

\hfill

\textbf{Пример:}
\begin{enumerate}
    \item[1)] $D = \{ A \}$, $A$ - точка
    \item[2)] $D = \{ AB \}$, $AB$ - отрезок
    \item[3)] Спрямленная (с конечной длиной) кривая
\end{enumerate}

\begin{theorem}
    Пусть $D$ - замкнутая плоская область. Тогда $D$ - квадрируемая граница области $\Delta$. $\Leftrightarrow$ имеет площадь 0. $\blacksquare$
\end{theorem}

\hfill

\begin{theorem}
    Пусть $\alpha$ - плоская спрямленная кривая. Тогда $\alpha$ - имеет площадь нуль. $\blacksquare$
\end{theorem}

\hfill

\underline{Следствие} Пусть 1) $D$ - область на плоскости, 2) $D$ ограничена конечным числом cпрямленных кривых. Тогда $D$ - квадрируема.

\hfill

\begin{note}
    В дальнейшем мы будем рассматривать только квадрируемые области
\end{note}

\section{Задачи, приводящие к понятию двойного интеграла}

\begin{enumerate}
    \item[I.] \textbf{Задача об объеме цилиндрического тела}

    Пусть $D$ - область плоскости $Oxy$

    $f$: $D \to R$ - функция определенная на множестве $D$

    $$f(x, y) \geq 0\ \ \ (x, y) \in D$$

    Рассмотрим тело $T$, которое ограничено плоскостью $Oxy$, графиком функции $z = f(x,y)$ и цилиндрической поверхностью, направляющая которой совпадает с гранью $D$, а образующие параллельны $Oz$

        \begin{enumerate}
        \item[1)] Разобьем область $D$ на пересекающиеся части

        $$ D = \overset{n}{\underset{i = 1}{U}} D_i $$

        $$ \text{int } D_i \cap \text{int } D_j = \emptyset, \text{ при } i \ne j\ (*)$$

        $$ \text{int } D_j - \text{множество внутренних точек области } D_i $$

        Условие (*) означает, что различные элементы разбиения не имеют общих внутренних точек

        \item[2)] Выберем точку $M_i \in D_i$ $i = \overline{1;n}$

        \item[3)] Считая, что размеры подобласти $D_i$ малы, примем $\Delta V_i \approx f(M_i) \Delta S_i$, где $\Delta S_i = S(D_i)$, $\Delta V_i$ - объем той части тела $T$, которая рассматривается под $D_i$
        \end{enumerate}

    \textbf{Тогда объем тела $T$:}
    
    $$ V = \sum_{i=1}^n \Delta V_i \approx \sum_{i=1}^n f(M_i) \Delta S_i $$
    
    Эта формула тем точнее, чем меньше размеры $D_i$, поэтому естественно перейти к пределу
    
    $$ V = \lim_{\underset{i = \overline{1,n}}{\text{max diam}}(D_i) \to 0} \sum_{i=1}^n f(M_i) \Delta S_i,$$
    
    $$\text{diam} (D) = \underset{M,N \in D}{\text{sup}} || \overline{MN} || - \text{диаметр множества } D$$
    
    \item[II.] \textbf{Задача о вычислении массы пластины}
    
    Пусть: 
    
    \begin{enumerate}
        \item[1)] Пластина занимает область $D$ на плоскости
        \item[2)] $f(x,y) \geq 0$ - плоскость поверхности материала пластины в точке $M(x,y)$
    \end{enumerate}
    
    Нужно найти массу $m$ этой частички
\end{enumerate}

\begin{enumerate}
    \item[1)] Разобьем область $D$ на непересекающей части
    
    $$ D = \overset{n}{\underset{i=1}{U}} D_i $$
    
    $$ \text{int } D_i \cap \text{int } D_j = \emptyset, i \ne j $$
    
    \item[2)] В пределах $D_i$ выберем точку $M_i, i = \overline{1,n}$
    \item[3)] Считая, что размеры $D_i$ малы, можно принять, что в пределах каждой из оластей $D_i$ плотность пластины меняется незначительно, поэтому во всех точках области $D_i$ плотность $\approx f(M_i)$ 

    Тогда масса части $D_i:\ \Delta m_i \approx f(M_i) \Delta S_i$, где $\Delta S_i = S(D_i),\ i = \overline{1, m}$

    \item[4)] \textbf{Тогда масса всей пластины}
    
    $$ m = \sum_{i=1}^n \Delta m_i \approx \sum_{i=1}^n f(M_i) \Delta S_i $$

\end{enumerate}

Полученная формула тем точнее, чем меньше размеры $D_i$, поэтому собственно

$$ m = \lim_{\underset{i = \overline{1,n}}{\text{max diam}} (D_i) \to 0} \sum_{i=1}^n f(M_i) \Delta S_i $$

\section{Определение свойства двойного интеграла}

Пусть $D$ - квадратичная замкнутая плоская область

\begin{defenition}
    \textbf{Разбиение области $D$} называется множество $R = \{ D_1, ... D_n\}$, где 

\begin{enumerate}
    \item[1)] $D = \underset{i=1}{\overset{n}{U}} D_i$
    \item[2)] $\text{int } D_i \cap \text{int } D_j = \emptyset$, при $i \ne j$
    \item[3)] $D_i$ - квадрируема, $i = \overline{1,n}$
\end{enumerate}
\end{defenition}

\begin{defenition}
\textbf{Диаметром разбиения} $R = \{ D_1, ... D_n \}$ называется число 

$$d(R) = \underset{i = \overline{1,n}}{\text{max diam}} (D_i)$$

\end{defenition}

Пусть $D$ - квадратичная замкнутая область на плоскости $Oxy$, $f: D \to R$ ($f$ является функцией двух переменных, т.к. $D$ - область на плоскости)

\begin{defenition}
    \textbf{Двойным интегралом функции $f$ по области $D$} называется число 

$$ \iint_D f(x,y) dx dy = \lim_{d(R) \to 0} \sum f(n_i) \Delta S_i, \text{ где}$$

$$ R = \{D_1,...D_n\} \text{ - разбиение области } D $$
$$ M_i \in D_i,\ i = \overline{1,n} \text{ - отмеченные точки} $$
$$ \Delta S_i = S(D_i) $$
\end{defenition}

\begin{defenition}
    В определении подразумевается, что указанный предел существует, конечен и не зависит от разбиения $R$ области $D$ и способа выбора отмеченных точек
\end{defenition}

\begin{defenition}
    Функции $f$, для которых существует $\iint_\Delta f dx dy$, называются \textbf{интегрируемыми в $D$}
\end{defenition}

\hfill

\textbf{Свойства двойного интеграла:}

\begin{enumerate}
    \item[1)] $\iint_D 1 dxdy = S(D)$
    \item[2)] Линейность 
    
    Если $f,g$ - интегрируемы в $D$ функции, то 
    
    \begin{enumerate}
        \item[а)] $f \pm g$ интегрируема в $D$, $\iint_D (f \pm g) dxdy = \iint_D fdxdy \pm \iint_D g dxdy$
        \item[б)] $c \cdot f, c = \text{const}$ - интегрируема, $\iint_D c \cdot f dxdy = c \iint_D f dxdy$
    \end{enumerate}
    
    \item[3)] Аддитивность
    
    Пусть
    
    \begin{enumerate}
        \item[1.] $D_1,D_2$ - плоские квадратичные области
        \item[2.] $f$ интегрируема в $D_1$ и $D_2$
        \item[3.] $\text{int } D_1 \cap \text{int } D_2 = \emptyset$
    \end{enumerate}
    
    Тогда $f$ интегрируема в $D= D_1 \cup D_2$
    
    $$ \iint_D f dxdy = \iint_{D_1} f dxdy + \iint_{D_2} f dxdy $$
    
    \item[4)] О сохранении интегралом знака функции
    
    Пусть
    
    \begin{enumerate}
        \item[1.] $F(x,y) \geq 0$ в $D$
        \item[2.] $f$ - интегрируема в $D$
    \end{enumerate}
    
    тогда 
    
    $$ \iint_D F(x,y) dxdy \geq 0 $$
    
    \item[5)] Пусть
    
        \begin{enumerate}
        \item[1.] $ f(x,y) \geq g(x,y) $
        \item[2.] $ f,g $ - интегрируемы в $D$
    \end{enumerate}
    
    тогда
    
    $$ \iint_D f dxdy \geq \iint_D g dxdy $$
    
    \item[6)] Теорема об оценке модуля двойного интеграла
    
    Пусть $f$ интегрируема в $D$, тогда $|f|$ - интегрируем в $D$
    
    $$ |\iint_D f dxdy| \le \iint_D |f| dxdy $$
    
    \item[7)] Теорема об оценке двойного интеграла (обобщенная теорема)
    
    Пусть 
    
    \begin{enumerate}
        \item[1.] $f,g$ - интегрируемы в $D$
        \item[2.] $m \le f(x,y) \le M$
        \item[3.] $g(x,y) \geq 0$
    \end{enumerate}
    
    тогда
    
    $$ m \iint_D g(x,y) dxdy \le \iint_D f(x,y) dxdy \le M \iint_D g(x,y) dxdy $$
    
    \underline{Следствие} Если $g(x,y) \equiv 1$ в $D$, то получаем "просто" теорема об оценке двойного интеграла
    
    $$ m \cdot S \le \iint_D f(x,y) dxdy \le M \cdot S, \text{ где} S = S(D)$$
    
    
    \item[8)] Теорема о среднем значении
    
        \begin{defenition}
            \textbf{Средним значением функции $f$ в плоскости $D$} называется
        \end{defenition}
    
    $$ <f> = {1 \over S(D)} \iint_D f(x,y) dxdy $$
    
    \underline{Свойство} Пусть 
    
    \begin{enumerate}
        \item[1.] $D$ - линейно связная замкнутая область (т.е. граница $D$ является связным множеством)
        \item[2.] $f$ - непрерывна в $D$
    \end{enumerate}
    
    Тогда существует $M_0 \in D$, такая что $f(M_0) = <f>$
    
    \item[9)] Обобщенная теорема о среднем значении
    
    Пусть
    
    \begin{enumerate}
        \item[1.] $f$ - непрерывна в $D$
        \item[2.] $g$ - интегрируема в $D$
        \item[3.] $g$ - знакопостоянна
        \item[4.] $D$ - линейно связанной множество (если $f$ - непрерывна в $D$, то $f$ - интегрируема в $D$)
    \end{enumerate}
    
    тогда существует $M_0 \in D$ такая, что 
    
    $$ \iint_d f(x,y)g(x,y) dxdy = f(M_0) \iint_d g(x,y) dxdy $$
    
\end{enumerate}

\underline{Замечание} Свойство (8) является частным случаем свойства (9) для $g(x,y) = 1$

\section{Повторный интеграл}

\begin{defenition} 
    Повторным интегралом называется выражение $ \int_a^b dx \int_{\varphi_1(x)}^{\varphi_2(x)} f(x,y) dy $, значение $I_\text{повт}$ которого определяется правилом $ I_\text{повт} = \int_a^b F(x) dx $, где $F(x) = \int_{\varphi_1(x)}^{\varphi_2(x)} f(x,y) dy$, $x \in [a,b], x = \text{const}$
\end{defenition}

Вычислить 
$$
I_\text{повт} = \int_1^{\ln(2)} dx \int_1^{1 \over x} x e^{xy} dy
$$

$$
\text{a)} F(x) = \int_1^{1 \over x} xe^{xy} dy = e^{xy} | _{y = 1}^{y = \frac{1}{x}} = e - e^x
$$

$$
\text{б)} I_\text{повт} = \int_1^{\ln(2)} F(x) dx = \int_1^{\ln(2)} (e - e^x)dx = e(\ln(2) - 1) - e^x |_1^{\ln(2)} = e \ln(2) - 2
$$

\section{Вычисление двойного интеграла}

\begin{defenition}
    Область $D$ на плоскости $Oxy$ называется $y$ - прав., если любая прямая, параллельная $Oy$, пересекает границу $D$ не более, чем в двух точках, либо содержит участок границы области $D$ целиком
\end{defenition}

\begin{note}
    \begin{enumerate}
        \item $y$-прав. можно задать в следующем виде:
        $$
            D = \{ (x,y) : a \le x \le b, \varphi_1(x) \le y \le \varphi)2(x) \}
        $$

    \item $x$ - прав. определеяется аналогично
    \end{enumerate}
\end{note}

\begin{theorem}
    Пусть 

    \begin{enumerate}
        \item $\exists \iint_D f(x,y) dxdy = I$
        \item $D$ является $y$-прав. и задается соотношением (*)
        \item $\forall x \in [a;b] \exists \int_{\varphi_1(x)}^{\varphi_2(x)} f(x,y) dy = F(x)$
    \end{enumerate}

    Тогда 

    \begin{enumerate}
        \item $\exists$ повторный интеграл

            $$
            \int^b_a \int_{\varphi_1(x)}^{\varphi_2(x)} f(x,y) = I_\text{повт}
            $$

        \item $I = I_\text{повт}$
    \end{enumerate}
\end{theorem}

\begin{note}
    Если область $D$ не является правильной в направлении какой-нибудь из координатных осей, то ее можно разбить на правильные части и воспользоваться свойством аддитивности двойного интеграла
\end{note}

\section{Замена переменных в двойном интеграле}

Пусть 

\begin{enumerate}
    \item $I = \iint _{D_{xy}} f(x,y) dxdy$
    \item $\varphi : D_{uv} \to D_{xy}$

        $$
        \varphi : \begin{cases}
            x = x(u,v) \\
            y = y(u,v)
        \end{cases}
        $$
\end{enumerate}

\begin{theorem}
    О замене переменных в двойном инетеграле

    Пусть

    \begin{enumerate}
        \item $D_{xy} = \varphi(D_{uv})$
        \item $\varphi$ биективно
        \item $\varphi$ непрерывна и непрерывано дифф. в $D_{uv}$
        \item $I_\varphi \ne 0$ в $D_{uv}$, где 
            $$
                I_\varphi = \left| \begin{matrix} x'_u & x'_v \\ y'_u & y'_v \end{matrix} \right|
            $$
        \item $f$ - интегрируема в $D_{xy}$
    \end{enumerate}
    
    Тогда

    \begin{enumerate}
        \item $f(x(u,v), y(u,v)) |I_\varphi (u,v)|$ - истина в $D_{uv}$
        \item $\iint_{D_{xy}} f(x,y) dxdy = \iint_{D_{uv}} f(x(u,v), y(y,v)) \cdot |I_\varphi (u,v)| dudv$
    \end{enumerate}
\end{theorem}

\begin{note}
    \begin{enumerate}
        \item Теорема остается справедливой и в том случае, если условия 2,3,4 нарушаются в отдельных точках области $D_{uv}$ или вдоль отдельных кривых, лежащих в $D_{uv}$ и имеющих площадь нуль
    \end{enumerate}
\end{note}

\section{Приложения двойного инетграла}

\begin{enumerate}
    \item[I.] \textbf{Вычисление площади плоской фигуры}

    $$
        S(D) = \iint_D 1 dx dy
    $$

    \item[II.] \textbf{Вычисление массы пластины}

        Пусть 
        \begin{enumerate}
            \item[1)] Пластина занимает обалсть $D$ на плоскости $Oxy$
            \item[2)] $f(x,y)$ - значение плотности материала пластины
        \end{enumerate}

        Тогда масса пластины

        $$
        M = \iint_D f(x,y) dxdy
        $$

    \item[III.] \textbf{Вычисление оъема тела}

        Пусть

        \begin{enumerate}
            \item[1)] Тело $T$: $T = \{ (x,y,z): (x,y) \in D_{xy}, z_1(x,y) \le z \le z_2(x,y) \}$
        \end{enumerate}

        Тогда объем тела $T$ можно найти по формуле

        $$
        V(T) = \iint_{D_{xy}} \big[ z_2(x,y) - z_1(x,y) \big] dxdy
        $$

\end{enumerate}

\chapter{Тройной интеграл}

\section{Понятие кубируемой области}

Рассмотрим область $G \subseteq R^3$

Как ввести понятие объема тела, которое занимает эту область? Понятие объема легко ввести для параллелепипеда или, более общо, многогранника в $R^3$. Что делать, если $G \subseteq R^3$ - произвольная область?

\begin{enumerate}
    \item Рассмотрим множество многогранников $q$, целиком содеожащихся в $G$, и обозначим 

        $$
        V_* = \underset{q}{\text{sup}} V(q)
        $$

    \item Рассмотрим множество многогранников $Q$, целико содержащих в себе $G$, и обозначим 

        $$
        V^* = \underset{Q}{inf} V(Q)
        $$
\end{enumerate}

\begin{defenition}
    Трехмерная область $G$ называется кубируемой, если $\exists$ конечные значения $V_*, V^*$, причем $V_* = V^*$. При этом значение $V = V_* = V^*$ называется бъемом области $G$
\end{defenition}

\begin{defenition}
    Говорят, что множество точек в $R^3$ имеет объем нуль, если все точки этого множества можно заключить в многогранник сколь угодно малого объема.
\end{defenition}

\section{Задача о вычислении массы тела}

Пусть 

\begin{enumerate}
    \item Тело $T$ занимает область $G \subset R^3$
    \item $f(x,y,z) \geq 0$ - значение плотности материала этого тела в точке $(x,y,z)$
\end{enumerate}

Требуется: Найти массу $m(T)$ тела $T$

\begin{enumerate}
    \item Разобьем область $G$ на части:

        $$
        G = U_{i=1}^{n} G_i,\ \text{int }G_i \cap \text{int } G_j = 0, \text{ при } i \ne j
        $$

    \item В пределах кажддой из подобластей выберем отмеченную точку $M_i \in G_i,\ i = \overline{1;n}$

    \item Считая, что размеры $G_i$ малы:

        $$
        \Delta m_i = m(G_i) \approx f(M_i) \Delta V_i, \text{ где } \Delta V_i = V(G_i)
        $$

        масса тела, занимающего подобласть $G_i$

    \item Масса тела $T$ тогда:

        $$
        m(T) = \sum_{i=1}^n \Delta m_i \approx \sum_{i=1}^n f(M_i) \Delta V_i
        $$

    \item Эта формула тем точнее, чем меньше размеры $G_i$, поэтому естественно перейти к пределу:

        $$
        m(T) = \underset{ \underset{i}{\text{max}}\text{ diam} \to 0 }{\lim} \sum_{i=1}^n f(M_i) \Delta V_i
        $$
\end{enumerate}

\section{Определение тройного интеграла}

Пусть 

\begin{enumerate}
    \item $G \subseteq R^3$ - тело
    \item $f:G \to R$ - функция
\end{enumerate}

Разоьем область $G$ на части так, как это было сделано в задаче о вычислении массы тела

Обозначим: $R = \{G_1,...,G_n\}$ - разбиение тела $G$

\begin{defenition}
    Диаметром разбиения $R$ тела $G$ называется число

    $$
    d(R) = \underset{i = \overline{1;n}}{\text{max diam }} G_i
    $$
\end{defenition}

\begin{defenition}
    Тройным интегралом по функции $f(x,y,z)$ по области $G$ называется число 

    $$
    \iiint_G f(x,y,z) dxdydz = \lim_{d(R) \to 0} \sum_{i=1}^n f(M_i) \Delta V_i
    $$

    , где $M_i, \Delta V_i$ имееют тот же смысл, что и в задаче о вычислении массы тела
\end{defenition}

\begin{note}
    Если указанный в определении тройного интеграла предел $\exists$ и конечен, то функция $f$ называется интегрируемой в области 
\end{note}

\subsection{Свойства тройного интеграла}

Эти свойства полностью аналогичны свойствам 1 - 9 двойного интеграла; при их записи нужно вместо $f(x,y) \mapsto f(x,y,z)$, $\iint_D f(x,y) dxdy \mapsto \iiint_G f(x,y,z) dxdydz$ $D \mapsto G$.

\section{Вычисление тройного интеграла}

Основная идея - сведение к повторному интегралу

\begin{defenition}
    Область $G \subseteq R^3$ называется z-правильной, если любая прямая, параллельная $Oz$, пересекает границу $G$ не более чем в двух точках или содержит участок границы целиком.
\end{defenition}

z-правильная область $G$ можно задать в виде:

$$
G = \{ (x,y,z) : (x,y) \in D_{xy}, z_1(x,y) \le z \le Z_2(x,y) \}
$$

\begin{theorem}
    Пусть

    \begin{enumerate}
        \item $\exists \iiint_G f(x,y,z) dxdydz$
        \item $G$ является z-прав и задается (*)
        \item Для каждой фиксированной точки $(x,y) \in D_{xy}$
    \end{enumerate}

    $$
    \exists \int_{z_1(x,y)}^{z_2(x,y)} f(x,y,z) dz = F(x,y)
    $$

    Тогда 

    \begin{enumerate}
        \item $\exists$ повторный интеграл

            $$
            I_\text{повт} = \iint_{D_{xy}} F(x,y) dxdy = \iint_{D_{xy}} dxdy \int_{z_1(x,y)}^{z_2(x,y)} f(x,y,z) dz
            $$

        \item и $I_\text{повт} = I$
    \end{enumerate}
\end{theorem}

\begin{note}
    Если в условии * сформулированной теоремы область $D_{xy}$ является y-правильной и задается в виде:

    $$
    D_{xy} = \{ (x,y): a \le x \le b, \varphi_1(x) \le y \le \varphi_2(x) \}
    $$

    то

    $$
    \iiint_G f(x,y,z) dxdyd = \int_a^bdx \int_{\varphi_1(x)}^{\varphi_2(x)} dy \int_{z_1(x,y)}^{z_2(x,y)} f(x,y,z) dz
    $$
\end{note}

\section{Замена переменных в тройном итеграле}

\begin{theorem}
    Пусть

    \begin{enumerate}
        \item $G_{xyz} = \varphi(G_{uv\omega})$
        \item $\varphi: G_{uv\omega} \to G_{xyz}$

            $$
            \varphi : 
            \begin{cases}
                x = x(u, v, \omega) \\
                y = y(u, v, \omega) \\
                z = z(u, v, \omega) \\
            \end{cases}
            $$

        \item Отображение $\varphi$ биективно
        \item $\varphi$ непрерывно и непрырывно дифференцируемо в $G_{uv\omega}$
        \item 
            $$
            J_\varphi(u,v,\omega) = \left|
            \begin{matrix}
                x'_u & x'_v & x'_\omega \\
                y'_u & y'_v & y'_\omega \\
                z'_u & z'_v & z'_\omega \\
            \end{matrix}
            \right| \ne 0
            $$
        \item $f(x,y,z)$ интегрируема в $G_{xyz}$
    \end{enumerate}

    Тогда

    $$
    \iiint_{G_{xyz}} f(x,y,z)dxdydz = \iiint_{G_{uv\omega}} f(x(u,v,\omega), y(u,v,\omega), z(u,v,\omega)) |J_\varphi(u,v,\omega)| dudvd\omega
    $$
\end{theorem}

\subsection{Связь цилиндрической и декартовой СК}

$$
\begin{cases}
    x = \rho \cos(\varphi) \\
    y = \rho \sin(\varphi) \\
    z = z \\
\end{cases}
$$

$$
J_\text{цил} = \left|
\begin{matrix}
    \cos(\varphi) & -\rho \sin(\varphi) & 0 \\
    \sin(\varphi) & \rho \cos(\varphi) & 0 \\
    0 & 0 & 1
\end{matrix}
\right| = \rho
$$

\subsection{Связь сферической и декартовой СК}

$$
\begin{cases}
    x = r \cos(\Theta) \cos(\varphi) \\
    y = r \cos(\Theta) \sin(\varphi) \\
    z = r \sin(\Theta) \\
\end{cases}
$$

$$
|J_\text{сф}| = \dots = r^2 \cos(\Theta)
$$

\chapter{Определения вероятности}

\section{Случайный эксперимент}

\begin{defenition}
    Случайным называется эксперимент, результат которого невозможно предсказать.
\end{defenition}

\begin{enumerate}
    \item Подброс монетки

        $$
        \Omega = \{ \text{Г}, \text{Р}\}
        $$

        $$
        |\Omega| = 2
        $$

    \item Бросают игральную кость

        $$
        \Omega = \{ 1, 2, 3, 4, 5, 6 \}
        $$

        $$
        |\Omega| = 6
        $$

    \item Бросают монету до первого появления герба

        $$
        \Omega = \{ 1, 2, 3, \dots \}
        $$

        $$
        |\Omega| = \aleph_0
        $$

        Омега является счетным множеством, т.е. в нем столько же элементов, сколько существует натуральных чисел.

    \item Производят стрельбу по плоской мишени размеры которой 1м x 1м (координаты - точки попадания)

        $$
        \Omega = \{ (x,y) : |x| \le \frac{1}{2}; |y| \le \frac{1}{2} \}
        $$

        $$
        |\Omega| = c
        $$

        Омега имеет можность континуума
\end{enumerate}

\begin{defenition}
    Множество $\Omega$ всех исходов данного случайного эксперимента называется пространством элементов исхода
\end{defenition}

\begin{note}
    При рассматривании пространства элементов исходов предполагается, что
    \begin{enumerate}
        \item Каждый элемент исхода неделим, т.е. не может быть "разложен" на более мелкие исходы
        \item В результате случайного эксперимента всегда происходит ровно один элемент исхода из $\Omega$
    \end{enumerate}
\end{note}

\begin{defenition}
    (Нестрогое) Событием называется (любое) подмножество множества $\Omega$
\end{defenition}

\begin{defenition}
    Говорят, что в результате случайного эксперимента происходит событие $A$, если в результате этого эксперимента произошел один из входящих в $A$ элементов исхода.
\end{defenition}

Бросают игральную кость

$$
\Omega = \{ 1,2,3,4,5,6 \}
$$

$$
A = \{2,4,6\}
$$

Если выпало 2 очко, то наступило $A$

\begin{defenition}
    Событие $A$ называется следствием события $B$, если наступление события $B$ влечет наступление события $A$, т.е. $B \subseteq A$
\end{defenition}

\begin{note}
    Любое множество $\Omega$ содержит в качестве подмножеств $\emptyset$ и $\Omega$ соответствующие события называются невозможным ($\emptyset$) и достоверным ($\Omega$). Оба этих события называют несобственными. Все остальные события называют собственными.
\end{note}

В урне находится 2 красных и 3 синих шара. Из урны извлекают 1 шар

$$
A = \{ \text{извлеченный шар - зеленый} \} = \emptyset
$$

$$
B = \{ \text{извлеченный шар - красный или синий} \} = \Omega
$$

\section{Операции над событиями}

События - множества (подмножества множества $\Omega$) $\Rightarrow$ $\cup, \cap, \overline{a}, \backslash, \Delta$ 

\begin{defenition}
    Суммой событий $A,B \subseteq \Omega$ называют событие

    $$
    A + B = A \cup B
    $$
\end{defenition}

\begin{defenition}
    Произведением событий $A,B \subseteq \Omega$ называют событие

    $$
    A \cdot B = A \cap B
    $$
\end{defenition}

\begin{defenition}
    $A \backslash B$ называется разностью событий $A$ и $B$
\end{defenition}

\begin{defenition}
    $\overline{A}$ называется событием, противоположным $A$
\end{defenition}

\subsection{Свойства операций над событиями}

Смотреть теоретико-множеств. тождества (осно.)

\begin{defenition}
    События $A,B \in \Omega$ называются несовместными, если $AB = \emptyset$. В противоположном случае события $A$ и $B$ называются совместными.
\end{defenition}

\begin{defenition}
    События $A_1, \dots, A_n, \dots$ называются попарнонесовместимимы, если $A_i A_j = \emptyset, i \ne j$ - несовместимыми в совокупности $A_1 \cdot \dots \cdot A_n = \emptyset$
\end{defenition}

\section{Классическое определение вероятности}

Пусть

\begin{enumerate}
    \item $|\Omega| = N < \infty$
    \item по условиям сложности эксперимента нет оснований предпочесть тот или иной исход остальных (в этом случае говорят, что все элементы исхода равновозможны)
    \item $A \subseteq \Omega,\ |A| = N_A$
\end{enumerate}

\begin{defenition}
    Вероятностью осуществления события $A$ называют число

    $$
    P\{A\} = \frac{N_A}{N}
    $$
\end{defenition}

2 раза бросают игральную кость

$$
A = \{ \text{сумма выпавших очков} \}
$$

$$
P(A) = ?
$$

Решение:

Исход: $(x_1, x_2)$, где $x_i$ - количество выпавших при $i$-м броске

$$
\Omega = \{ (1,1), (1,2), \dots, (6,6) \}
$$

$$
|\Omega| = 36 = N
$$

$$
A = \{ (5,6), (6,5), (6,6) \}
$$

$$
N_A = |A| = 3 \\
$$

$$
P(A) = \frac{N_A}{N} = \frac{3}{36} = \frac{1}{12}
$$

\subsection{Свойства вероятности (в соответствии с классическим определением)}

\begin{enumerate}
    \item $P(A) \geq 0$
    \item $P(\Omega) = 1$
    \item Если $AB = \emptyset$, то $P(A+B) = P(A) + P(B)$
\end{enumerate}

\begin{proof}
    \begin{enumerate}
        \item $P(A) = \frac{N_A}{N} \geq 0$
        \item $P(\Omega) = \frac{N_\Omega}{N} = \frac{N}{N} = 1$
        \item $|A+B|=|A|+|B|-|AB|$ (формула включений и исключений). 
            
            По условию $|AB|=0 \Rightarrow N_{A+B} = N_A + N_B$

            $$
            P(A) = \frac{N_{A+B}}{N} = \frac{N_A + N_B}{N} = \frac{N_A+N_B}{N}=\frac{N_A}{N} + \frac{N_B}{N} = P(A) + P(B)
            $$
    \end{enumerate}
\end{proof}

\begin{note}
    Недостатки классического определения вероятности:

    \begin{enumerate}
        \item Неприменимо в случае, когда $|\Omega| = \infty$
        \item Неприменимо, если вектор исхода является "более возможным", чем другие
    \end{enumerate}
\end{note}

\section{Геометрическое определенеие вероятности}

является обобщением классического опредеения на случай бесконечного $\Omega$, когда $\Omega \subseteq R^n$

Пусть

\begin{enumerate}
    \item $\Omega \subseteq R^n$
    \item $\mu(\Omega) < \infty$, где мю - мера множества ($n=1$ - длина, $n=2$ - площадь)
    \item Возможность принадлежности исхода эксперимента некоторого события пропорциональна мере этого события и не язависит от его (события) формы и расположения внутри $\Omega$.
\end{enumerate}

\begin{defenition}
    Вероятностью осуществления события $A$ называется число

    $$
    P\{A\} = \frac{\mu(A)}{\mu(\Omega)}
    $$
\end{defenition}

Задача о встрече

Два человека договорились встретиться в условленном месте в промежутке от 12 до 14 часов. При этом если один из них придет раньше другого, то он ждет 15 минут, после чего уходит. Какова вероятность того, что они встретятся, если появления каждого из них равновероятно в любой момент между 12 и 13 часами?

Решение

\begin{enumerate}
    \item Исход

        $$
        (x_1,x_2)
        $$

        где $x_i \in [0,1], i = 1,2$ - появление i-го человека после 12 часов

        $$
        \Omega = \{ (x_1,x_2) : x_i \in [0;1]\} = [0;1] \times [0;1]
        $$

    \item $A = \{\text{эти два человека встретились}\}$

        $$
        A = \{ (x_1,x_2) : |x_1-x_2| \le \frac{1}{4}\}
        $$

    \item В соотвествии с геометрическим определением

        $$
        P(A) = \frac{\mu(A)}{\mu(\Omega)} = \frac{\mu(\Omega) - 2 \mu(\Delta)}{\mu(\Omega)} = 1 - 2 \frac{1}{2} (\frac{3}{4})^2 = 1 - \frac{9}{16} = \frac{7}{16}
        $$
\end{enumerate}

\begin{note}
    \begin{enumerate}
        \item Очевидно, что из геометрического определения следуют те же свойства вероятности, что и из классического определения
        \item Недостатком геометрического определения является то, что оно не учитывает возможность того, что некоторые области внутри $\Omega$ могут быть более предпочтительными, чем другие области той же меры. Например, если в разобранном примере появление каждого из этих двух человек было более вероятным в середине часа, то геометрическое определение дало бы неудовлетворительный результат.
    \end{enumerate}
\end{note}

\section{Статистическое определение вероятности}

Пусть 

\begin{enumerate}
    \item $\Omega$ пространство элементарных исходов случайного эксперимента
    \item $A \subseteq \Omega$ - событие, связанное с этим экспериментом
    \item Этот случайным эксперимент произведен $n$ раз, при этом событие $A$ произошло $n_A$ раз
\end{enumerate}

\begin{defenition}
    Вероятностью события $A$ называется эмпирический (то есть из опыта) предел:

    $$
    P(A) = \lim_{n \to \infty} \frac{n_A}{n}
    $$
\end{defenition}

\begin{note}
    \begin{enumerate}
        \item Из статистического определения можно поучить те же свойства вероятности, что и из двух предыдущих определений
        \item Недостатки статистического определения

            \begin{itemize}
                \item Никакой эксперимент невозможно осуществить бесконечное чтсло раз
                \item С точки зрения современной математики это определение является архаизмом, так как не дает достаточной базы для развития теории
            \end{itemize}
    \end{enumerate}
\end{note}

\section{Сигма-алгебра событий}

Для аксиматического определения вероятности необходимо уточнить понятие события

Заметим, что:

\begin{itemize}
    \item данное ранее нестрогое определение события как произврольного подмножества в $\Omega$ использовать нельзя, так как в этом случае теория будет противоречивой (смотреть парадокс Рассела)
    \item по этой причине событиями мы будем называть лишь те подмножества множества $\Omega$, которые принадлежат заранее оговоренному набору подмножеств
    \item с точки зрения здравого смысла понятно, что если относительно событий $A$ и $B$ известно, наступили они в данном эксперименте или нет, то также должно быть известно, наступили ли в этом эксперименте события $\overline{A}$, $A+B$, $AB$, ... По этой причине указанный набор подмножеств должен быть замкнут относительно операций над событиями $\overline{}, +, \cdot, \backslash$ ... Эти соображения приводят к следующему определению
\end{itemize}

\begin{defenition}
    Пусть
    \begin{enumerate}
        \item $\Omega$ - пространство 
        \item $B$ - набор подмножеств множества $\Omega$

            $B$ называется $\sigma$-алгеброй событий, если
            \begin{enumerate}
                \item $B \ne \emptyset$
                \item $A \in B \Rightarrow \overline{A} \in B$
            \end{enumerate}

        \item Если $A_1, \dots, A_n, \dots \in B$, то $A_1 + \dots + A_n + \dots \in B$
    \end{enumerate}
\end{defenition}

\subsection{Простейшие свойства сигма-алгебры событий}

\begin{enumerate}
    \item $\Omega \in B$
    \item $\emptyset \in B$
    \item если $A_1, \dots, A_n, \dots \in B$, то $A_1 \cdot \dots \cdot A_n \cdot \dots \in B$
    \item если $A,B \in B$, то $A \backslash B \in B$
\end{enumerate}

\begin{proof}
    \begin{enumerate}
        \item $B \ne \emptyset \Rightarrow \exists A \in B$

            В соотвествии с аксиомой 2) $\overline{A} \in B$

            В соответствии с 3) $\underbrace{A + \overline{A}}_{\Omega} \in B$

        \item $\Omega \in B \Rightarrow \overline{\Omega} \in B$

        \item $\overline{A_1} + \dots + \overline{A_n} + \dots \in B \Rightarrow \overline{\overline{A_1} + \dots + \overline{A_n} + \dots} \in B \Rightarrow A_1 \cdot \dots \cdot A_n \cdot \dots \in B$

        \item $A \backslash B = A \overline{B}$

            $A, B \in B \Rightarrow a, \overline{B} \in B \Rightarrow A \overline{B} \in B$
    \end{enumerate}
\end{proof}

\begin{note}
    \begin{enumerate}
        \item В дальнейшем, говоря о вероятности всегда будем предполагать, что задана некоторая сигма-алгебра событий. При этом слово "событие" всегда будет обозначать элемент этой сигма-алгебры
        \item Если множество $\Omega$ конечно, то в качестве сигма-алгебры событий на $\Omega$ всегда будем рассматривать

            $$
            B = 2^\Omega
            $$
    \end{enumerate}
\end{note}

Случайно выбранного человека попросили выбрать одно из трех: камень, ножныци, бумагу

$$
\Omega = \{\text{К,Н,Б}\}
$$

$$
B = 2^\Omega = \{ \emptyset, \{\text{К}\}, \{\text{Н}\}, \{\text{Б}\}, \{\text{К,Н}\}, \{\text{К,Б}\}, \{\text{Н,Б}\}, \underbrace{\{\text{К,Н,Б}\}}_\Omega \}
$$

\section{Аксиоматическое определение вероятности}

Пусть

\begin{enumerate}
    \item $\Omega$ пространство элементов исходов некторого эксперимента
    \item $B$ - сигма-алгебра на $\Omega$
\end{enumerate}

\begin{defenition}
    Вероятностью (вероятностной мерой) называется отображение

    $$
    P: B \to R
    $$

    обладающее свойствами

    \begin{enumerate}
        \item $\forall A \in B P(A) \geq 0$ (аксиома неотрицательности)
        \item $P(\Omega) = 1$ (аксиома нормированности)
        \item если $A_1, ..., A_n, ... \in B$ попарно несовместные события, то $P(A_1 + .. + A_n + ..) = P(A_1) + ... + P(A_n) + ...$ (расширенная аксиома сложения)
    \end{enumerate}
\end{defenition}

\begin{defenition}
    Тройка $(\Omega, B, P)$ называется вероятностным пространством
\end{defenition}

\subsection{Свойства вероятности}

\begin{enumerate}
    \item $P(\overline{A}) = 1 - P(A)$
    \item $P(\emptyset) = 0$
    \item Если $A \subseteq B$, то $P(A) \le P(B)$
    \item $\forall A \in B 0 \le P(A) \le 1$
    \item $P(A+B)=P(A)+P(B)-P(AB)$
    \item Для любого конечного набора событий $A_1,...A_n \in B$ справедливо 

        $$
        P(A_1+...+A_n) = \sum_{1 \le i \le n} P(A_i) - \sum_{1 \le i_1 < i_2 \le n}P(A_{i_1}, A_{i_2}) + \sum_{1 \le i_1 < i_2 < i_3 \le n} P(A_{i_1}, A_{i_2}, A_{i_3}) - ... + (-1)^{n+1} P(A_1\cdot ... \cdot A_n)
        $$
\end{enumerate}

\begin{proof}
    \begin{enumerate}
        \item $A + \overline{A} = \Omega$

            $A \overline{A} = \emptyset \Rightarrow \text{ аксиома 3 } P(A + \overline{A}) = P(A) + P(\overline{A})$

            $\Rightarrow P(\overline{A}) = 1 - P(A)$

        \item $P(\emptyset) = P(\overline{\Omega}) = \text{ свойство 1 } = 1 - P(\Omega) = 0$

        \item $A \subseteq B$

            $B = A + (B \backslash A) \Rightarrow P(B) = P(A) + P(B \backslash A)$

            $\Rightarrow P(B) \geq P(A)$

        \item $P(A) \geq 0$ вытекает из аксиомы 1

            Покажем, что $P(A) \le 1$

            $A \subseteq \Omega \Rightarrow \text{ по свойсту}$
    \end{enumerate}
\end{proof}

%
%
%
% ДОПИСАТЬ ВНУТРИ PROOF
%
%

\begin{note}
    Иногда вместо расширенной аксиомы сложения 3 рассматривают следующие две аксиомы

    \begin{enumerate}
        \item[3')] Для любых попарно несовместимых событий $A_1,...$ $A_n P(A_1...+A_n) = P(A_1) + ... + P(A_n)$ (аксиома сложений)

        \item[3'')] Для любых несовместимых поселедовательностей событий $A_1 \subseteq A_2 \subseteq ... \subseteq A_n \subseteq ...$ справедливо

            $$
            \lim_{n \to \infty} P(A_n) = P(A), \text{ где } A = A_1 + ... A_n + ...
            $$

            (аксиома непрерывности)
    \end{enumerate}

    Можно показать, что

    $$
    3^o \Leftrightarrow
    \begin{cases}
        3' \\
        3'' \\
    \end{cases}
    $$
\end{note}

\chapter{Условная вероятность}

\section{Определение условной вероятности}

Пусть 

\begin{enumerate}
    \item $A, B$ - случайные события, связанные с некоторым экспериментом
    \item известно, что в результате эксперимента произошло событие $B$ 
\end{enumerate}

Как эта информация повлияет на вероятность того, что в результате этого эксперимента произошло событие $A$?

Из колоды в 36 карт случайным образом извлекли одну карту

$$
A = \{ \text{извлечен туз} \}
$$

$$
B = \{ \text{извлечена картинка} \}
$$

Тогда

$$
P(A) = \frac{4}{36} = \frac{1}{9}
$$

$$
P_B(A) = \text{наступило B} \to \text{извлечена карта B,D,K,T} = \frac{4}{16} = \frac{1}{4}
$$

Таким образом дополнительная информация об осуществлении события $B$ изменила вероятность события $A$

\begin{note}
    Рассмотрим классическую схему для определения вероятности имеется $N$ развовозможных исходов, $|A| = N_A,\ |B| = N_B$

    Так как известно, что в результате эксперимента наступило $B$, то вне исхода, не попавшие $B$, можно не рассматривать

    В этом случае событие $A$ может наступить лишь при реализации одного из исходов, входящих в $AB$.

    $$
    P_B(A) = \frac{N_{AB}}{N_B} = \frac{N_{AB} / N}{N_B / N} = \frac{P(AB)}{A(B)}
    $$
\end{note}

\begin{defenition}
    Пусть $(\Omega, \beta, P)$ - вероятностное пространство (не обязательно реализует классическую схему)

    Условная вероятность осуществления события $A$ при условии, что произошло событие $B$, называют число

    $$
    P(A|B) = \frac{P(AB)}{P(B)}, P(B) \ne 0
    $$
\end{defenition}

\begin{note}
    \begin{enumerate}
        \item Для того, чтобы подчеркнуть разницу, "обычную" вероятность иногда будем называть безусловной
        \item Зафиксируем некоторое событие $B$ и будем рассматривать $P(A|B)$ как функцию события $A \in \beta$
    \end{enumerate}
\end{note}

\begin{theorem}
    Условная вероятность $P(A|B)$ удовлетворяет трем аксиомам безусловной вероятности 
\end{theorem}

\begin{proof}
    \begin{enumerate}
        \item 
            $$
            P(A|B) = \frac{P(AB) \geq 0}{P(B) > 0} \geq 0
            $$

        \item
            $$
            P(\Omega|B) = \frac{P(\Omega B)}{P(B)} = \frac{P(B)}{P(B)} = 1
            $$

        \item пусть $A_1, ..., A_n,...$ - набор попарнонепесекающихся событий

            $$
            P(A_1 + A_2 + ... | B) = \frac{P((A_1 + A_2 + ...)B)}{P(B)} = \text{свойство счетной дистрибутивности} = \frac{P(A_1B + A_2B + ...)}{P(B)}
            $$

            \begin{enumerate}
                \item $A_i \cdot A_j = \emptyset \text{ при } i \ne j$
                \item $A_iB \subseteq A_j$
                \item а,б $\Rightarrow (A_i, B)(A_j, B) = \emptyset \Rightarrow$ расширенная аксиома сложения для $A_1B, A_2B, ...$
            \end{enumerate}

            $$
            = \frac{P(A_1B) + P(A_2B) + ...}{P(B)} = \text{множества сход. рядов} = 
            $$

            $$
            = \frac{P(A_1B)}{P(B)} + \frac{P(A_2B)}{P(B)} + ... = P(A_1|B) + P(A_2|B) + ...
            $$
    \end{enumerate} 
\end{proof}

\begin{cont}
    Условная вероятность $P(A|B)$ при фиксированном событии $B$ обладает всеми свойствами безусловной вероятности:

    \begin{enumerate}
        \item $P(\overline{A}|B) = 1 - P(A|B)$
        \item $P(\emptyset|B) = 0$
        \item Если $A_1 \subseteq A_2$, то $P(A_1|B) \le P(A_2|B)$
        \item $0 \le P(A|B) \le 1$
        \item $P(A_2 + A_2|B) = P(A_1|B) + P(A_2|B) - P(A_1A_2|B)$
        \item $P(A_1+...A_n|B) = \sum_{1\le i_1 \le n} P(A_{i_1}|B) = \sum_{1 \le i_1 < i_2 \le n} P(A_{i_1}A_{i_2}|B) + \sum_{1 \le i_1 < i_2 < i_3 \le n} A(A_{i_1}A_{i_2}A_{i_3}|B) + ... + (-1)^{n-1} P(A_1A_2...A_n|B)$
    \end{enumerate}
\end{cont}

\begin{proof}
    Свойства 1-6 безусловой вероятности ялвяются следствияеми аксиом 1-3 вроятности. Так как условная вероятность удовлетворяет этим аксиомам, то для нее выполнятся и аналоги свойств 1-6.
\end{proof}

Среды 15 лотерийных билетов 5 выйгрышных. Сначала 1-й игрок тянет 1 билет, затем 2-й тянет один билет.

$$
A_1 = \{ \text{первый игрок достал выйгрышный билет} \}
$$

$$
A_2 = \{ \text{второй игрок достал выйгрышный билет} \}
$$

$$
P(A_2|A_1) = ?
$$

1-й способ по определению условной вероятности

$$
P(A_2|A_1) = \frac{P(A_1A_2)}{P(A_1)}
$$

\begin{enumerate}
    \item Исход $(x_1,x_2)$, где $x_i$ - номер билета, извлеченного 2-м игроком, $x_i \in \{ 1, ... , 15 \}$ - размещение без повторов из 15 по 2

        $$
        N = 15 \cdot 14
        $$

        $$
        (\underbrace{x_1}_{\text{выигр.}}, ?)
        $$

    \item $N_A = 5 \cdot 14$

        $$
        P(A_1) = \frac{5 \cdot 14}{15 \cdot 14} = \frac{1}{3}
        $$

    \item $N_{A_1A_2} = 5 \cdot 4 = 20$

        $$
        (\underbrace{x_1}_5, \underbrace{x_2}_4)
        $$

        $$
        P(A_1A_2) = \frac{20}{15 \cdot 14} = \frac{2}{21}
        $$

    \item $P(A_2|A_1)$

        $$
        P(A_2 | A_1) = \frac{P(A_1A_2)}{P(A_1)} = \frac{2}{7}
        $$
\end{enumerate}

2-й способ подсчитаем $P(A_2|A_1)$, перестроив в пространство $\Omega$

$$
P(A_2|A_1) = \text{известно, что наступило } A_1 \Rightarrow \text{ осталось 14 билетов, из кот. 4 выигр.} = \frac{4}{14} = \frac{2}{7}
$$

\section{Формула умножения вероятностей}

\begin{theorem}
    Формула умножения вероятностей для двух событий

    Пусть 

    \begin{enumerate}
        \item $A_1, A_2$ - слбытия связанные с некоторым случайным экспериментом
        \item $P(A_1) > 0$
    \end{enumerate}

    Тогда

    $$
    P(A_1A_2) = P(A_1)P(A_2|A_1)
    $$
\end{theorem}

\begin{proof}
    \begin{enumerate}
        \item Так как $P(A_1) \ne 0$, то по определению

            $$
            P(A_2|A_1) = \frac{P(A_1A_2}{P(A_1)} \Rightarrow P(A_1A_2) = P(A_1)P(A_2|A_1)
            $$
    \end{enumerate}    
\end{proof}

\begin{theorem}
    Формула умножения вероятностей для $n$ событий

    Пусть

    \begin{enumerate}
        \item $A_1,...,A_n$ - события, связанные с некоторым случайным экспериментом
        \item $P(A_1,...,A_n) > 0$
    \end{enumerate}

    Тогда

    $$
    P(A_1 \cdot A_2 ... A_n) = P(A_1)P(A_2|A_1)P(A_3|A_1A_2) \cdot ... \cdot P(A_n|A_1...A_{n-1})
    $$
\end{theorem}

\begin{proof}
    \begin{enumerate}
        \item $A_1 \cdot ... \cdot A_{n-1} \subseteq A_1 \cdot ... \cdot A_k$, если $k \le n-1$

            $\Rightarrow P(A_1 \cdot ... \cdot A_{n-1}) > 0$

            Таким образом все входящие в правую часть формулы умножения условной вероятности определены

        \item $P(\underbrace{A_1 \cdot ... \cdot A_{n-1}}_A, \underbrace{A_n}_B) = $ из теоремы умножения для 2-х событий $=$

            $$
            = P(A_1 \cdot ... \cdot A_{n-1})P(A_n|A_1 \cdot ... \cdot A_{n-1}) =
            $$

            $$
            = P(A-1 \cdot ... \cdot A_{n-2})P(A_{n-1}|A_1 \cdot ... \cdot A_{n-2})P(A_n | A_1 \cdot ... \cdot A_{n-1}) = ... =
            $$

            $$
            = P(A_1)P(A_2|A_1)P(A_3|A_1A_2) \cdot ... \cdot P(A_n|A_1...A_{n-1})
            $$
    \end{enumerate}
\end{proof}

На 7 карточках написаны буквы, составляющие слово "шоколад". Карточки перемешивают и случайным образом извлекают последовательно 3 карточки (без возвращения)

$$
A = \{ \text{в порядке извлечения эти карточки образуют слово "код".} \}
$$

\begin{enumerate}
    \item Обозначим

        $$
        A_1 = \{ \text{при первом извлечении появилась "к".} \}
        $$

        $$
        A_2 = \{ \text{при втором извлечении появилась "о".} \}
        $$

        $$
        A_3 = \{ \text{при третьем извлечении появилась "д".} \}
        $$

        Тогда

        $$
        A = A_1 A_2 A_3
        $$

    \item 
        $$
        P(A) = P(A_1A_2A_3) = \underbrace{P(A_1)}_{\frac{1}{7}} \underbrace{P(A_2|A_1)}_{\frac{2}{6} = \frac{1}{3}} \underbrace{P(A_3|A_1A_2)}_{\frac{1}{5}} = \frac{1}{105}
        $$

\end{enumerate}

\section{Независимые события}

\begin{defenition}
    Пусть $A$ и $B$ - события, связанные с некоторым случайным экспериментом. События $A$ и $B$ называются независимыми, если

    $$
    P(AB) = P(A)P(B)
    $$
\end{defenition}

\begin{theorem}
    \hfill

    \begin{enumerate}
        \item Пусть $P(B) > 0$,

            Тогда $A,B$ - независимые $\Leftrightarrow P(A|B = P(A)$

        \item Пусть $P(A) > 0$,

            Тогда $A,B$ - независимые $\Leftrightarrow P(B|A) = P(B)$
    \end{enumerate}
\end{theorem}

\begin{proof}
    Докажем первую часть

    \begin{enumerate}
        \item $\Rightarrow$ (необходимость)

            $$
            P(B|A) = \frac{P(BA)}{P(B)} = \text{события независимы} = \frac{P(A)P(B)}{P(B)} = P(A)
            $$

        \item $\Leftarrow$ (достаточность)

            $$
            P(AB) = P(B) > 0 \Rightarrow \text{ используем теорему умножения вероятностей} = P(B) \cdot P(A|B) = 
            $$

            $$
            = \text{по условию P(A|B) = P(A)} = P(A)P(B) \Rightarrow A,B \text{ - независимы}
            $$
    \end{enumerate}
\end{proof}

\begin{note}
    В качестве определения независимых событий $A$ и $B$ кажется более логичным выбрать условие$P(A|B) = P(A) \text{ или } P(B|A) = P(B)$, а не условие  $P(AB) = P(A)P(B)$. Однако последнее условие работает всегда, а то время как первые два условия работают лишь при $P(B) > 0\ (P(A) > 0)$
\end{note}

Из колоды 36 карт случайным образом извлекают одну карту.

$$
A = \{ \text{извлечен туз} \}
$$

$$
B = \{ \text{извлечена карта красной масти} \}
$$

Являются ли $A$ и $B$ независимыми

\begin{enumerate}
    \item $P(A) = \frac{4}{36} = \frac{1}{9}$
    \item $P(B) = \frac{18}{36} = \frac{1}{2}$
    \item $P(AB) = | AB = \{ \text{извлечен туз красной масти} \} | = \frac{2}{36} = \frac{1}{18}$
    \item $P(AB) = P(A)P(B)$

        $$
        \frac{1}{18} = \frac{1}{9} \cdot \frac{1}{2}
        $$

        Верно $\Rightarrow$ $A$,$B$ - независ.
\end{enumerate}

\begin{theorem}
    Пусть $A$,$B$ - независимые. Тогда независимыми являются события

    \begin{enumerate}
        \item $\overline A$ и $B$
        \item $A$ и $\overline B$
        \item $\overline A$ и $\overline B$
    \end{enumerate}
\end{theorem}

\begin{proof}
    \begin{enumerate}
        \item Проверим равенство $P(\overline A B) = P(\overline A) P(B)$
        
            \begin{enumerate}
                \item если $P(B) = 0 \Rightarrow$ Пр. 4 = 0

                $$
                \overline A B \subseteq B \Rightarrow P(\overline A B) \le P(B) = 0 \Rightarrow P(\overline A B) = 0
                $$

            \item если $P(B) > 0$, то

                $$
                P(\overline A B) = P(B) P(\overline A | B) = P(B)(1-P(A|B)) = \text{A,B - независимые}
                $$

                $$
                = P(B)(1-P(A)) = P(\overline A)P(B)
                $$
            \end{enumerate}

        \item Аналогично доказать самостоятельно
        \item Аналогично доказать самостоятельно
    \end{enumerate}
\end{proof}

\begin{defenition}
    События $A_1,...,A_n$ незываются независимыми попарно, если

    $$
    \forall \forall i,j \in \{1,...,n\},i \ne j, A_i \text{ и } A_j \text{ - независимые}
    $$
\end{defenition}

\begin{defenition}
    События $A_1,...A_n$ называются независимыми в совокупности, если

    $$
    \forall k \in \{2,...,n\} \forall \forall i_1,...,i_k : 1 \le i_1 < i_2 < ... < i_k \le n
    $$

    $$
    P(A_{i_1},...,A_{i_k}) = P(A_{i_1}) \cdot ... \cdot P(A_{i_k})
    $$
\end{defenition}

\begin{note}
    Это определение означает, что $A_1,...,A_n$ - независимы в совокупности, если:

    \begin{enumerate}
        \item $P(A_{i_1} A_{i_2}) = P(A_{i_1})P(A_{i_2})$
        \item $P(A_{i_1} P_{i_2} P_{i_3}) = P(A_{i_1})P(A_{i_2}) P(A_{i_3})$
        \item[...]
        \item[n-1.] $ P(A_1 ... A_n) = P(A_1) \cdot ... \cdot P(A_n) $
    \end{enumerate}
\end{note}

\begin{note}
    Очевидно, что $A_1,...,A_n$ - независимы в совокупности $\Rightarrow$ $A_1,...,A_n$ - независимы попарно, обратное неверно.
\end{note}

Рассмотрим правильный тетраэдр, на гранях которого написаны цифры 1,2,3. ПРичем на первой грани написана только 1, на второй написано 2, на третьей 3, а на последней написаны все 3 цифры.

Тэтраэдр подбрасывают.

$$
A_1 = \{ \text{на нижней грани 1} \}
$$

$$
A_2 = \{ \text{на нижней грани 2} \}
$$

$$
A_3 = \{ \text{на нижней грани 3} \}
$$

$$
P(A_1) = \frac{2}{4} = \frac{1}{2} = P(A_2) = P(A_3)
$$

$$
P(A_1A_2) = | \text{на нижней грани есть 1 и 2} | = \frac{1}{4} = P(A_1A_3) = P(A_2A_3)
$$

Таким образом

$$
P(A_1A_2) = P(A_1)P(A_2)
$$


$$
P(A_1A_3) = P(A_1)P(A_3)
$$

$$
P(A_2A_3) = P(A_2)P(A_3)
$$

$$
P(A_1A_2A_3) = |\text{на нижней грани одновременно 1,2,3}| = \frac{1}{4} \ne \frac{1}{8} = P(A_1)P(A_2)P(A_3)
$$

Таким образом $A_1,A_2,A_3$ не являются независимыми в совокупности

\section{Формула полной вероятности}

Пусть $(\Omega, \beta, P)$ - вероятностное пространство

\begin{defenition}
    Будем говорить, что события $H_1, ..., H_n$ образуют полную группу, если

    \begin{enumerate}
        \item $H_1 + ... + H_n = \Omega$
        \item $H_i H_j = \emptyset, i\ne j$
        \item $P(H_i) > 0, i = \overline{1;n}$
    \end{enumerate}
\end{defenition}

\begin{theorem}
    О формуле полной вероятности

    Пусть

    \begin{enumerate}
        \item $H_1,...,H_n$ - полная группа событий
        \item $A \in B$ - некоторое событие
    \end{enumerate}

    Тогда

    $$
    P(A) = P(A|H_1)P(H_1) + ... + P(A|H_n)P(H_n)
    $$
\end{theorem}

\begin{proof}
   Пусть $A$ может захыватывать некоторые события из $H_1,H_2,...,H_n$

   $$
   P(A) = P(A \Omega) = P(A(H_1+...+H_n)) = P(AH_1 + ... + AH_n) = 
   $$

   События $H_iH_j = \emptyset$, $(AH_i) \subseteq H_i, (AH_j) \subseteq H_j  \Rightarrow (AH_j)(AH_i) = \emptyset$

   $$
   = P(AH_1) + .. + P(AH_n) = P(A|H_1)P(H_1) + ... + P(A|H_n)P(H_n)
   $$
\end{proof}

В магазин поступили телевизоры 3-х фирм, из которых 30\% произведено 1-й фирмой, 50\% произведено второй фирмой, 20\% произведено 3-й фирмой. Известно, что среди телевизоров 1-й фирмы 7\%, 2-й - 5\%, 3-й - 10\% брака. Найти вероятность того, что случайно выбранный телевизор окажется бракованным?

\begin{enumerate}
    \item Рассмотрим полную группу событий

        $$
        H_1 = \{ \text{выбранный телевизор произведен 1-й фирмой} \}
        $$


        $$
        H_2 = \{ \text{выбранный телевизор произведен 2-й фирмой} \}
        $$


        $$
        H_3 = \{ \text{выбранный телевизор произведен 3-й фирмой} \}
        $$

        Обозначим:

        $$
        A = \{ \text{выбранный телевизор бракованный} \}
        $$

    \item Формула полной вероятности

        $$
        P(A) = \underbrace{P(A|H_1)}_{0.07} \underbrace{P(H_1)}_{0.3} + \underbrace{P(A|H_1)}_{0.05} \underbrace{P(H_2)}_{0.5} + \underbrace{P(A|H_3)}_{0.1} \underbrace{P(H_3)}_{0.2} = ... = 0.066
        $$
\end{enumerate}

\section{Формула Байеса}

\begin{theorem}
    О формуле Байеса

    Пусть

    \begin{enumerate}
        \item Выполнены условия теоремы о формуле полной вероятности
        \item $P(A) > 0$
    \end{enumerate}

    Тогда

    $$
    P(H_i|A) = \frac{P(A|H_i) P(H_i)}{P(A|H_1) P(H_1) + ... + P(A|H_n) P(H_n) }, i = \overline{1;n}
    $$
\end{theorem}

\begin{proof}
    $$
    P(H_i|A) = \frac{P(A H_i)}{P(A)} = \frac{P(A|H_i) P(H_i)}{P(A|H_1) P(H_1) + ... + P(A|H_n) P(H_n) }
    $$
\end{proof}

Рассмотрим пример о покупке телевизора. Пусть известно, что куплен бракованный телевизор. Какой фирмой он вероятнее всего произведен?

$$
P(H_1| \underbrace{A}_\text{куплен брак. тел.}) = \frac{P(A|H_1)P(H_1)}{P(A)} = \frac{0.07 \cdot 0.3}{P(A)} = \frac{0.021}{P(A)}
$$

$$
P(H_2|A) = \frac{P(A|H_2)P(H_2)}{P(A)} = \frac{0.05 \cdot 0.5}{P(A)} = \frac{0.025}{P(A)} \text{ - max}
$$

$$
P(H_3|A) = \frac{P(A|H_3)P(H_3)}{P(A)} = \frac{0.2 \cdot 0.1}{P(A)} = \frac{0.02}{P(A)}
$$

Ответ: вероятнее всего этот телевизор произведен второй фирмой

\begin{note}
    \begin{enumerate}
        \item События $H_1,...,H_n$, образующие полную группу, часто называют \textbf{гипотезами}
        \item Вероятности $P(H_i), i = \overline{1;n}$ - называют \textbf{априорными}, так как они известны до опыта. Вероятности $P(H_i|A), i = \overline{1;n}$, которые становятся известны только после эксперимена, называют \textbf{апостериорными}.
    \end{enumerate}
\end{note}

\section{Схема испытаний Бернулли}

Рассмотрим случайный эксперимент, в результате которого возможна реализация одного из двух элементарных исходов, условно называемых успехом и неудачей, то есть в рассматриваемом случайном эксперимента

$$
\Omega = \{ 0,1 \}, \text{ где 0 - неудача, а 1 - успех }
$$

Обозначим $P\{\text{успех}\} = p$, тогда $P\{\text{неудача}\} = 1 - p = q$

\begin{enumerate}
    \item Подбрасвают монету, успех - выпадает герб, неудача - выпадение решки
    \item Бросают игральную кость, успех - выпадение 6, неудача - все остальное
    \item Наблюдает пол новорожденного, успех - рождение мальчика, неудача - рождение девочки
\end{enumerate}

\begin{defenition}
    Схемой испытаний Бернулли будем называть серию однотипных экспериментов указанного вида, в которой вероятность реализации успеха не изменяется от эксперимента к эксперименту.
\end{defenition}

\begin{note}
    Условие неизменности вероятности успеха на протяжении всей серии означает, что отдельные испытания независмиы. Другими словами, вероятность реализации успеха в $j$-м эксперименте не зависит от исходов, имевших место в 1-м, 2-м, ..., $j-1$-м испытаниях.
\end{note}

Обозначим $P_n(k)$ - вероятность того, что в серии из $n$ экспериментов по схеме Бернулли прошло ровно $k$ успехов

\begin{theorem}
    Пусть проводится серия из $n$ экспериментов по схеме Бернулли с вероятностью успеха $p$. Тогда

    $$
    P_n(k) = C_n^k p^k q^{n-k}, \text{ где } C_n^k = \frac{n!}{k!(n-k)!} \text{ - биномиальные коэффициенты}, q = 1-p, k = \overline{0;n}
    $$
\end{theorem}

\begin{proof}
    \hfill

    \begin{enumerate}
        \item Запишем результат проведения серии из $n$ экспериментов с использованием кортежа

            $$
            (x_1,x_2,...,x_n), \text{ где } x_i =
            \begin{cases}
                1, \text{ если в i-м исходе успех} \\
                0, \text{ если в i-м исходе неудача} \\
            \end{cases}
            $$

            $$
            A = \{ \text{в серии из n экспериментов произошло ровно k успехов} \} = \{ (x_1,... x_n) : \sum_{i=1}^n x_i = k \}
            $$

        \item $|A| = ?$

            Каждый входящий в $A$ кортеж однозначно определеяется номерами позиций, в которых стоят единицы, то есть набором $k$ чисел из множества $\{ 1,.., n\}$. Таких наборов существует $C_n^k$ штук, то есть $|A| = C_n^k$

            $$
            (0,1,0,0,1,1), n = 6
            $$

        \item Рассмотрим исход $(x_1, ..., x_n) \in A$. Вероятность осуществления:

            $$
            P\{(x_1,...,x_n)\} = P \{ \{ \text{в 1-м испытании x1} \} \cdot \{ \text{во 2-м испытании x2} \} \cdot ... \cdot \{ \text{в n-м эксперименте xn} \} \} =
            $$

            $$
            = P\{ \text{} \} \cdot ... \cdot P\{ \text{} \} =
            $$

            В серии ровно $k$ успехов и $n-k$ неудач $\Rightarrow$ В этом  $k$ 

        \item Так как вероятность осуществления неудачного исхода из $A$ равна $p^k q^{n-k}$, а всего в $A$ $C_n^k$ исходов, то $P(A) = C_n^k p^k q^{n-k}$
    \end{enumerate}
\end{proof}

\begin{cont}
    $$
    P_N(k_1 \le k \le k_2) = \sum_{j=k_1}^{k_2} C_n^j p^j q^{n-j}
    $$

    Вероятность того, что число успехов в серии из $n$ экспериментов по схеме Бернулли заключено между $k_1$ и $k_2$
\end{cont}

\begin{proof}
    \hfill

    \begin{enumerate}
        \item Обозначим:

            $$
            A = \{ k_1 \le k \le k_2 \}
            $$

            $$
            A_j = \{ k = j \}, j = \overline{k_1;k_2}
            $$

            Тогда

            $$
            A = \sum_{j = k_1}^{k_2} A_j
            $$

        \item $A_j \cdot A_l = \{ \text{в серии произошло одновременно j и ровно l успехов} \} =$

            $$
            =
            \begin{cases}
                A_j, j=l \\
                0, j \ne l \\
            \end{cases}
            $$

            $$
            P(A) = P(\sum_{j=k_1}^{k_2}) = \{попарное пересечение Aj\} = \sum_{j=k_1}^{k_2} P(A_j) = \sum_{j=k_1}^{k_2} C_n^j p^j q^{n-j}
            $$
    \end{enumerate}
\end{proof}

\begin{cont}
    $$
    P_n(k \geq 1) = 1 - q^n
    $$

    Вероятность того, что в серии из $n$ экспериментов по схеме Бернулли произошел хотя бы один успех
\end{cont}

\begin{proof}
    Пусть $A = \{ \text{в серии произошле хотя бы один успех} \}, \overline A = \{ \text{ни одного успеха} \}$

    $$
    P(A) = 1 - P(\overline{A}) = 1 - P_n(0) = 1 - C_n^0 p^0 q^n = 1 - q^n
    $$
\end{proof}

5 раз бросают игральную кость.

$$
A = \{ \text{6 выпадет ровно два раза} \}
$$

$$
B = \{ \text{6 выпадет хотя бы 2 раза} \}
$$

$$
P(A),P(B) = ?
$$

Успех = \{выпадение 6\}

Неудача = \{ выпадение 1,2,3,4,5 \}

$$
p = \frac{1}{6}\ \ q = \frac{5}{6}
$$

\begin{enumerate}
    \item
        $$
        P(A) = P_5(2) = C_5^2 \cdot \bigg(\frac{1}{6}\bigg)^2 \cdot \bigg(\frac{5}{6}\bigg)^3 \approx 0.161
        $$

    \item

        $$
        P(B) = P_5(2 \le k \le 5) = \sum_{j=2}^5 C_5^j \bigg(\frac{1}{6}\bigg)^j \bigg(\frac{5}{6}\bigg)^{5-j}
        $$

        $$
        = 1 - P(\overline B) = 1 - P_5(0 \le k \le 1) = 1 - P_5(0) - P_5(1) = 1 - \bigg(\frac{5}{6}\bigg)^5 - 5 \bigg(\frac{1}{6} \bigg) \bigg(\frac{5}{6}\bigg)^4 \approx 0.196
        $$
\end{enumerate}

СЛУЧАЙНЫЕ ВЕЛИЧИНЫ

\chapter{Одномерные случайные велечины}

\section{Понятие случайной величины}

\begin{defenition}
    (Нестрогое) Пусть исход некоторого случайного эксперимента можно описать числом $X$, тогда $X$ - случайная велечина
\end{defenition}

\begin{enumerate}
    \item Бросают монету

        $$
        X =
        \begin{cases}
            0, \text{ если выпала решка} \\
            1, \text{ если выпал герб}
        \end{cases}
        $$

    \item $n$ раз бросают игралюную кость

        $$
        X_1 - \text{число выпадений 6}, X_1 = \{0,1,2,...,n\}
        $$

        $$
        X_2 - \text{суммарное число выпавших очков}, X_2 \in \{n, n+1, ..., 6n\}
        $$

    \item У случайно выбранного пациента в больнице измеряют темературу $X$ тела, $X \in [34,42]$
\end{enumerate}

\begin{defenition}
    Пусть $(\Omega, \beta, P)$ - вероятностное пространство

    Случайной величиной называется отображение $X: \Omega \to R$ такое, что $\forall x \in R$ множество

    $$
    \{ \omega : X(\omega) < x \} \in \beta - \text{ является событием}
    $$
\end{defenition}

\begin{note}
    \hfill

    \begin{enumerate}
        \item Упрощенно на случайную величину моэно смотреть, как на случайный эксперимент, в котором бросают точку на прямую (случайным образом)
        \item Предположим, что мы провели эксперимент с бросанием точки достаточно большое число раз. Отложим в точках прямой частоты появления отдельных возможых значений случайной величины

            \begin{enumerate}
                \item Если

                    $$
                    X =
                    \begin{cases}
                        0, \text{ если решка} \\
                        1, \text{ если герб}
                    \end{cases}
                    $$

                    то частоты появления 0 и 1 будут примерно равны $\frac{1}{2}$

                \item Если $X_1 -$ число выпадений 6, то $\{ \lambda = 0\} = P_n(0) = \big( \frac{5}{6} \big)^n$

                \item $X - $ температура тела пациента
            \end{enumerate}

        \item Таким образом различные случайные величины могут иметь различные множества значений. При этом у различных случайных величин даже одному и тому же значению могту отвечать различные вероятности
    \end{enumerate}
\end{note}

\begin{defenition}
    Законом распределения верояности случайной величины называется правило, которое возможным значениям (множествам значений) этой случайной величины приписывает вероятности того, что она примет эти значения или значения из этих множеств.
\end{defenition}

Универсальным способом задания закона распределения любой случайной величины является задание ее функции распределения вероятностей.

\section{Функция распределения вероятностей}

\begin{defenition}
    Пусть $X$ - случайная величина

    Функцией распределения вероятностей случайной величины $X$ называется отображение: $F: R \to R$, определенное правилом $F: x \to P\{X < x\}$
\end{defenition}

2 раза бросают симметричную монету, $X$ - число выпадений герба. Найти функцию распределения случайной величины $X$

\begin{enumerate}
    \item $X \in \{0,1,2\}$

        $$F(x_1) = P\{X < x_1\} = 0$$

        $$F(0) = P\{X < 0\} = 0$$

        $$P\{X=0\} = P_2(0) = q^2 = \frac{1}{4}$$

        $$P\{X=1\} = P_2(1) = \frac{1}{2}$$

        $$P\{X=2\} = P_2(2) = \frac{1}{4}$$

        \begin{tabular}{|c||c|c|c|}
            \hline
            $X$ & 0 & 1 & 2 \\
            \hline
            $P$ & $\frac{1}{4}$ & $\frac{1}{2}$ & $\frac{1}{4}$ \\
            \hline
        \end{tabular}

        $$
        F(x_2) = P\{X<x_2\} = P\{X=0\} = \frac{1}{4}
        $$

        $$
        F(1) =  P\{X<1\} = P\{X=0\} = \frac{1}{4}
        $$

        $$
        F(x_2) =  P\{X<x_2\} = P\{X\in\{0,1\}\} = P\big\{ \{X=0\} + \{X=1\} \big\} = P\{X=0\} + P\{X=1\} = \frac{3}{4}
        $$

        Таким образом

        $$
        F(x) =
        \begin{cases}
            0, x \le 0 \\
            \frac{1}{4}, 0 < x \le 1 \\
            \frac{3}{4}, 1 < x \le 2 \\
            1, 2 < x
        \end{cases}
        $$
\end{enumerate}

\subsection{Свойства функции распределения}

\begin{enumerate}
    \item $0 \le F(x) \le 1$
    \item если $x_1 \le x_2$, то $F(x_1) \le F(x_2)$
    \item $\lim_{x \to - \infty} F(x) = 0$, $\lim_{x \to +\infty} F(x) = 1$
    \item $P\{x_1 \le X < x_2\} = F(x_2) - F(x_1)$
    \item $\lim_{x \to x_0} F(x) = F(x_0)$, то есть в каждой точки $x \in R$ функция распределения непрерывна слева
\end{enumerate}

\begin{proof}

    \hfill

    \begin{enumerate}
        \item $F(x) = P\{...\} \Rightarrow 0 \le F(x) \le 1$

        \item $\{x < x_2\} = \{X < x_1\} + \{x_1 \le X < x_2\} (*)$

            $$
            P\{X < x_2\} = P \big\{ \{X<x_1\} + \{x_1 \le X < x_2 \} \big\} = P\{X < x_1\} + P\{x_1 \le X < x_2\} \Rightarrow F(x_2) \geq F(x_1)
            $$

        \item (*) $\Rightarrow P\{x_1 \le X < x_2\} = F(x_2) - F(x_1)$
    \end{enumerate}
\end{proof}

\end{document}
