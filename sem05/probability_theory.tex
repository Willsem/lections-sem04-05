\documentclass[a4paper, 14pt]{report}

\usepackage{cmap}
\usepackage[T2A]{fontenc}
\usepackage[utf8]{inputenc}
\usepackage[english,russian]{babel}
\usepackage[left=30mm, top=20mm, right=20mm, bottom=20mm, nohead, nofoot]{geometry}
\usepackage{indentfirst}

\usepackage{amsmath}
\usepackage{MnSymbol}
\usepackage{wasysym}

\usepackage{amsthm}
\usepackage{mdframed}

\newmdtheoremenv{theorem}{Теорема}[chapter]
\newtheorem{lemma}{Лемма}[chapter]
\newtheorem{defenition}{Определение}[chapter]
\newtheorem{note}{Замечание}[chapter]

\author{Власов Павел Александрович}
\title{Теория вероятностей}
\date{2019}

\begin{document}
	\maketitle
	
	\tableofcontents
	\clearpage
	
	% ==================================================================
	% ==================================================================
	%					  ГЛАВА 1: ДВОЙНОЙ ИНТЕГРАЛ
	% ==================================================================
	% ==================================================================
	\chapter{Двойной интеграл}
	
	% ==================================================================
	%				  ЧАСТЬ 1: ПЛОЩАДЬ ПЛОСКОЙ ФИГУРЫ
	% ==================================================================
	\section{Площадь плоской фигуры}
	
	Пусть $D$ - фигура на плоскости.
	
	Как ввести понятие площади фигуры $D$?
	
	Если $D$ является треугольником (или прямоугольником) понятие площади очевидно.
	
	Если $D$ является многоугольником, то ее можно разбить на треугольники, а площадь области $D$ определить как сумму состовляющих ее треугольников.
	
	Что делать если $D$ - произвольная фигура
	
	\begin{enumerate}
	
		\item[а)] Рассмотрим множество многоугольников $M$, каждый из которых целиком содержится в $D$.
		
		Обозначим $S_* = \text{sup}\ S(m)$, где $m$ - многоугольники, $S(m)$ - площадь многоугольника
		
		\item[б)] Рассмотрим множество многоугольников $M$, каждый из которых содержит в себе $D$.
		
		Обозначим $S^* = \text{suf}\ S(M)$
	
	\end{enumerate}
	
    \begin{defenition}
        Область $D$ на плоскости называется \textbf{квадрируемой}, если $\exists$ конечные значения $S_*$, $S^*$ причем $S_* = S^*$. При этом число $S = S_* = S^*$ называется \textbf{площадью области $D$}
    \end{defenition}
	
	\hfill
	
    \begin{defenition}
        Говорят, что множество $D$ точек плоскости имеет площадь \textbf{нуль}, если $D$ можно целиком заключить в многоугольник, сколь угодной площади, т.е. $\forall \varepsilon > 0\ \exists$ многоугольник $M$ площади $\varepsilon$ такой, что $D \le M$
    \end{defenition}
	
	\hfill
	
	\textbf{Пример:}
	\begin{enumerate}
		\item[1)] $D = \{ A \}$, $A$ - точка
		\item[2)] $D = \{ AB \}$, $AB$ - отрезок
		\item[3)] Спрямленная (с конечной длиной) кривая
	\end{enumerate}
	
    \begin{theorem}
        Пусть $D$ - замкнутая плоская область. Тогда $D$ - квадрируемая граница области $\Delta$. $\Leftrightarrow$ имеет площадь 0. $\blacksquare$
    \end{theorem}
	
	\hfill
	
    \begin{theorem}
        Пусть $\alpha$ - плоская спрямленная кривая. Тогда $\alpha$ - имеет площадь нуль. $\blacksquare$
    \end{theorem}
	
	\hfill
	
	\underline{Следствие} Пусть 1) $D$ - область на плоскости, 2) $D$ ограничена конечным числом cпрямленных кривых. Тогда $D$ - квадрируема.
	
	\hfill
	
    \begin{note}
        В дальнейшем мы будем рассматривать только квадрируемые области
    \end{note}
	
	% ==================================================================
	%	  ЧАСТЬ 2: ЗАДАЧИ, ПРИВОДЯЩИЕ К ПОНЯТИЮ ДВОЙНОГО ИНТЕГРАЛА
	% ==================================================================
	\section{Задачи, приводящие к понятию двойного интеграла}
	
	\begin{enumerate}
		\item[I.] \textbf{Задача об объеме цилиндрического тела}
		
		Пусть $D$ - область плоскости $Oxy$
		
		$f$: $D \to R$ - функция определенная на множестве $D$
		
		$$f(x, y) \geq 0\ \ \ (x, y) \in D$$
		
		Рассмотрим тело $T$, которое ограничено плоскостью $Oxy$, графиком функции $z = f(x,y)$ и цилиндрической поверхностью, направляющая которой совпадает с гранью $D$, а образующие параллельны $Oz$
		
		 \begin{enumerate}
		 	\item[1)] Разобьем область $D$ на пересекающиеся части
			
			$$ D = \overset{n}{\underset{i = 1}{U}} D_i $$
			
			$$ \text{int } D_i \cap \text{int } D_j = \emptyset, \text{ при } i \ne j\ (*)$$
			
			$$ \text{int } D_j - \text{множество внутренних точек области } D_i $$
			
			Условие (*) означает, что различные элементы разбиения не имеют общих внутренних точек
			
			\item[2)] Выберем точку $M_i \in D_i$ $i = \overline{1;n}$
			
			\item[3)] Считая, что размеры подобласти $D_i$ малы, примем $\Delta V_i \approx f(M_i) \Delta S_i$, где $\Delta S_i = S(D_i)$, $\Delta V_i$ - объем той части тела $T$, которая рассматривается под $D_i$
		 \end{enumerate}
		 
		\textbf{Тогда объем тела $T$:}
		
		$$ V = \sum_{i=1}^n \Delta V_i \approx \sum_{i=1}^n f(M_i) \Delta S_i $$
		
		Эта формула тем точнее, чем меньше размеры $D_i$, поэтому естественно перейти к пределу
		
		$$ V = \lim_{\underset{i = \overline{1,n}}{\text{max diam}}(D_i) \to 0} \sum_{i=1}^n f(M_i) \Delta S_i,$$
		
		$$\text{diam} (D) = \underset{M,N \in D}{\text{sup}} || \overline{MN} || - \text{диаметр множества } D$$
		
		\item[II.] \textbf{Задача о вычислении массы пластины}
		
		Пусть: 
		
		\begin{enumerate}
			\item[1)] Пластина занимает область $D$ на плоскости
			\item[2)] $T(x,y) \geq 0$ - плоскость поверхности материала пластины в точке $M(x,y)$
		\end{enumerate}
		
		Нужно найти массу $m$ этой частички
	\end{enumerate}
	
	\begin{enumerate}
		\item[1)] Разобьем область $D$ на непересекающей части
		
		$$ D = \overset{n}{\underset{i=1}{U}} D_i $$
		
		$$ \text{int } D_i \cap \text{int } D_j = \emptyset, i \ne j $$
		
		\item[2)] В пределах $D_i$ выберем точку $M_i, i = \overline{1,n}$
		\item[3)] Считая, что размеры $D_i$ малы, можно принять, что в пределах каждой из оластей $D_i$ плотность пластины меняется незначительно, поэтому во всех точках области $D_i$ плотность $\approx f(M_i)$ 
	
		Тогда масса части $D_i:\ \Delta m_i \approx f(M_i) \Delta S_i$, где $\Delta S_i = S(D_i),\ i = \overline{1, m}$
	
		\item[4)] \textbf{Тогда масса всей пластины}
		
		$$ m = \sum_{i=1}^n \Delta m_i \approx \sum_{i=1}^n f(M_i) \Delta S_i $$
	
	\end{enumerate}
	
	Полученная формула тем точнее, чем меньше размеры $D_i$, поэтому собственно
	
	$$ m = \lim_{\underset{i = \overline{1,n}}{\text{max diam}} (D_i) \to 0} \sum_{i=1}^n f(M_i) \Delta S_i $$
	
	% ==================================================================
	%	  ЧАСТЬ 3: ОПРЕДЕЛЕНИЕ СВОЙСТВА ДВОЙНОГО ИНТЕГРАЛА
	% ==================================================================
	\section{Определение свойства двойного интеграла}
	
	Пусть $D$ - квадратичная замкнутая плоская область
	
    \begin{defenition}
        \textbf{Разбиение области $D$} называется множество $R = \{ D_1, ... D_n\}$, где 
	
	\begin{enumerate}
		\item[1)] $D = \underset{i=1}{\overset{n}{U}} D_i$
		\item[2)] $\text{int } D_i \cap \text{int } D_j = \emptyset$, при $i \ne j$
		\item[3)] $D_i$ - квадрируема, $i = \overline{1,n}$
	\end{enumerate}
    \end{defenition}
	
    \begin{defenition}
    \textbf{Диаметром разбиения} $R = \{ D_1, ... D_n \}$ называется число 
	
	$$d(R) = \underset{i = \overline{1,n}}{\text{max diam}} (D_i)$$

    \end{defenition}
	
	Пусть $D$ - квадратичная замкнутая область на плоскости $Oxy$, $f: D \to R$ ($f$ является функцией двух переменных, т.к. $D$ - область на плоскости)
	
    \begin{defenition}
        \textbf{Двойным интегралом функции $f$ по области $D$} называется число 
	
	$$ \iint_D f(x,y) dx dy = \lim_{d(R) \to 0} \sum f(n_i) \Delta S_i, \text{ где}$$
	
	$$ R = \{D_1,...D_n\} \text{ - разбиение области } D $$
	$$ M_i \in D_i,\ i = \overline{1,n} \text{ - отмеченные точки} $$
	$$ \Delta S_i = S(D_i) $$
    \end{defenition}
	
    \begin{defenition}
        В определении подразумевается, что указанный предел существует, конечен и не зависит от разбиения $R$ области $D$ и способа выбора отмеченных точек
    \end{defenition}
	
    \begin{defenition}
        Функции $f$, для которых существует $\iint_\Delta f dx dy$, называются \textbf{интегрируемыми в $D$}
    \end{defenition}
	
	\hfill
	
	\textbf{Свойства двойного интеграла:}
	
	\begin{enumerate}
		\item[1)] $\iint_D 1 dxdy = S(D)$
		\item[2)] Линейность 
		
		Если $f,g$ - интегрируемы в $D$ функции, то 
		
		\begin{enumerate}
			\item[а)] $f \pm g$ интегрируема в $D$, $\iint_D (f \pm g) dxdy = \iint_D fdxdy \pm \iint_D g dxdy$
			\item[б)] $c \cdot f, c = \text{const}$ - интегрируема, $\iint_D c \cdot f dxdy = c \iint_D f dxdy$
		\end{enumerate}
		
		\item[3)] Аддитивность
		
		Пусть
		
		\begin{enumerate}
			\item[1.] $D_1,D_2$ - плоские квадратичные области
			\item[2.] $f$ интегрируема в $D_1$ и $D_2$
			\item[3.] $\text{int } D_1 \cap \text{int } D_2 = \emptyset$
		\end{enumerate}
		
		Тогда $f$ интегрируема в $D= D_1 \cup D_2$
		
		$$ \iint_D f dxdy = \iint_{D_1} f dxdy + \iint_{D_2} f dxdy $$
		
		\item[4)] О сохранении интегралом знака функции
		
		Пусть
		
		\begin{enumerate}
			\item[1.] $F(x,y) \geq 0$ в $D$
			\item[2.] $f$ - интегрируема в $D$
		\end{enumerate}
		
		тогда 
		
		$$ \iint_D F(x,y) dxdy \geq 0 $$
		
		\item[5)] Пусть
		
			\begin{enumerate}
			\item[1.] $ f(x,y) \geq g(x,y) $
			\item[2.] $ f,g $ - интегрируемы в $D$
		\end{enumerate}
		
		тогда
		
		$$ \iint_D f dxdy \geq \iint_D g dxdy $$
		
		\item[6)] Теорема об оценке модуля двойного интеграла
		
		Пусть $f$ интегрируема в $D$, тогда $|f|$ - интегрируем в $D$
		
		$$ |\iint_D f dxdy| \le \iint_D |f| dxdy $$
		
		\item[7)] Теорема об оценке двойного интеграла (обобщенная теорема)
		
		Пусть 
		
		\begin{enumerate}
			\item[1.] $f,g$ - интегрируемы в $D$
			\item[2.] $m \le f(x,y) \le M$
			\item[3.] $g(x,y) \geq 0$
		\end{enumerate}
		
		тогда
		
		$$ m \iint_D g(x,y) dxdy \le \iint_D f(x,y) dxdy \le M \iint_D g(x,y) dxdy $$
		
		\underline{Следствие} Если $g(x,y) \equiv 1$ в $D$, то получаем "просто" теорема об оценке двойного интеграла
		
		$$ m \cdot S \le \iint_D f(x,y) dxdy \le M \cdot S, \text{ где} S = S(D)$$
		
		
		\item[8)] Теорема о среднем значении
		
            \begin{defenition}
                \textbf{Средним значением функции $f$ в плоскости $D$} называется
            \end{defenition}
		
		$$ <f> = {1 \over S(D)} \iint_D f(x,y) dxdy $$
		
		\underline{Свойство} Пусть 
		
		\begin{enumerate}
			\item[1.] $D$ - линейно связная замкнутая область (т.е. граница $D$ является связным множеством)
			\item[2.] $f$ - непрерывна в $D$
 		\end{enumerate}
		
		Тогда существует $M_0 \in D$, такая что $f(M_0) = <f>$
		
		\item[9)] Обобщенная теорема о среднем значении
		
		Пусть
		
		\begin{enumerate}
			\item[1.] $f$ - непрерывна в $D$
			\item[2.] $g$ - интегрируема в $D$
			\item[3.] $g$ - знакопостоянна
			\item[4.] $D$ - линейно связанной множество (если $f$ - непрерывна в $D$, то $f$ - интегрируема в $D$)
		\end{enumerate}
		
		тогда существует $M_0 \in D$ такая, что 
		
		$$ \iint_d f(x,y)g(x,y) dxdy = f(M_0) \iint_d g(x,y) dxdy $$
		
	\end{enumerate}

	\underline{Замечание} Свойство (8) является частным случаем свойства (9) для $g(x,y) = 1$
 
    \section{Повторный интеграл}

    \begin{defenition} 
        Повторным интегралом называется выражение $ \int_a^b dx \int_{\varphi_1(x)}^{\varphi_2(x)} f(x,y) dy $, значение $I_\text{повт}$ которого определяется правилом $ I_\text{повт} = \int_a^b F(x) dx $, где $F(x) = \int_{\varphi_1(x)}^{\varphi_2(x)} f(x,y) dy$, $x \in [a,b], x = \text{const}$
    \end{defenition}

    Вычислить 
    $$
    I_\text{повт} = \int_1^{\ln(2)} dx \int_1^{1 \over x} x e^{xy} dy
    $$

    $$
    \text{a)} F(x) = \int_1^{1 \over x} xe^{xy} dy = e^{xy} | _{y = 1}^{y = \frac{1}{x}} = e - e^x
    $$

    $$
    \text{б)} I_\text{повт} = \int_1^{\ln(2)} F(x) dx = \int_1^{\ln(2)} (e - e^x)dx = e(\ln(2) - 1) - e^x |_1^{\ln(2)} = e \ln(2) - 2
    $$

    \section{Вычисление двойного интеграла}

    \begin{defenition}
        Область $D$ на плоскости $Oxy$ называется $y$ - прав., если любая прямая, параллельная $Oy$, пересекает границу $D$ не более, чем в двух точках, либо содержит участок границы области $D$ целиком
    \end{defenition}

    \begin{note}
        \begin{enumerate}
            \item $y$-прав. можно задать в следующем виде:
            $$
                D = \{ (x,y) : a \le x \le b, \varphi_1(x) \le y \le \varphi)2(x) \}
            $$

        \item $x$ - прав. определеяется аналогично
        \end{enumerate}
    \end{note}

    \begin{theorem}
        Пусть 

        \begin{enumerate}
            \item $\exists \iint_D f(x,y) dxdy = I$
            \item $D$ является $y$-прав. и задается соотношением (*)
            \item $\forall x \in [a;b] \exists \int_{\varphi_1(x)}^{\varphi_2(x)} f(x,y) dy = F(x)$
        \end{enumerate}

        Тогда 

        \begin{enumerate}
            \item $\exists$ повторный интеграл

                $$
                \int^b_a \int_{\varphi_1(x)}^{\varphi_2(x)} f(x,y) = I_\text{повт}
                $$

            \item $I = I_\text{повт}$
        \end{enumerate}
    \end{theorem}

    \begin{note}
        Если область $D$ не является правильной в направлении какой-нибудь из координатных осей, то ее можно разбить на правильные части и воспользоваться свойством аддитивности двойного интеграла
    \end{note}

    \section{Замена переменных в двойном интеграле}

    Пусть 

    \begin{enumerate}
        \item $I = \iint _{D_{xy}} f(x,y) dxdy$
        \item $\varphi : D_{uv} \to D_{xy}$

            $$
            \varphi : \begin{cases}
                x = x(u,v) \\
                y = y(u,v)
            \end{cases}
            $$
    \end{enumerate}

    \begin{theorem}
        О замене переменных в двойном инетеграле

        Пусть

        \begin{enumerate}
            \item $D_{xy} = \varphi(D_{uv})$
            \item $\varphi$ биективно
            \item $\varphi$ непрерывна и непрерывано дифф. в $D_{uv}$
            \item $I_\varphi \ne 0$ в $D_{uv}$, где 
               % $$
                    %I_\varphi = \left| \begin{matix} x'_u & x'_v \\ y'_u & y'_v \end{matrix} \right|
               % $$
            \item $f$ - интегрируема в $D_{xy}$
        \end{enumerate}
        
        Тогда

        \begin{enumerate}
            \item $f(x(u,v), y(u,v)) |I_\varphi (u,v)|$ - истина в $D_{uv}$
            \item $\iint_{D_{xy}} f(x,y) dxdy = \iint_{D_{uv}} f(x(u,v), y(y,v)) \cdot |I_\varphi (u,v)| dudv$
        \end{enumerate}
    \end{theorem}

    \begin{note}
        \begin{enumerate}
            \item Теорема остается справедливой и в том случае, если условия 2,3,4 нарушаются в отдельных точках области $D_{uv}$ или вдоль отдельных кривых, лежащих в $D_{uv}$ и имеющих площадь нуль
        \end{enumerate}
    \end{note}

    \section{Приложения двойного инетграла}

    \begin{enumerate}
        \item[I.] \textbf{Вычисление площади плоской фигуры}

        $$
            S(D) = \iint_D 1 dx dy
        $$

        \item[II.] \textbf{Вычисление массы пластины}

            Пусть 
            \begin{enumerate}
                \item[1)] Пластина занимает обалсть $D$ на плоскости $Oxy$
                \item[2)] $f(x,y)$ - значение плотности материала пластины
            \end{enumerate}

            Тогда масса пластины

            $$
            M = \iint_D f(x,y) dxdy
            $$

        \item[III.] \textbf{Вычисление оъема тела}

            Пусть

            \begin{enumerate}
                \item[1)] Тело $T$: $T = \{ (x,y,z): (x,y) \in D_{xy}, z_1(x,y) \le z \le z_2(x,y) \}$
            \end{enumerate}

            Тогда объем тела $T$ можно найти по формуле

            $$
            V(T) = \iint_{D_{xy}} \big[ z_2(x,y) - z_1(x,y) \big] dxdy
            $$

    \end{enumerate}

    \chapter{Тройной интеграл}

    \section{Понятие кубируемой области}

    Рассмотрим область $G \subseteq R^3$

    Как ввести понятие объема тела, которое занимает эту область? Понятие объема легко ввести для параллелепипеда или, более общо, многогранника в $R^3$. Что делать, если $G \subseteq R^3$ - произвольная область?

    \begin{enumerate}
        \item Рассмотрим множество многогранников $q$, целиком содеожащихся в $G$, и обозначим 

            $$
            V_* = \underset{q}{\text{sup}} V(q)
            $$

        \item Рассмотрим множество многогранников $Q$, целико содержащих в себе $G$, и обозначим 

            $$
            V^* = \underset{Q}{inf} V(Q)
            $$
    \end{enumerate}

    \begin{defenition}
        Трехмерная область $G$ называется кубируемой, если $\exists$ конечные значения $V_*, V^*$, причем $V_* = V^*$. При этом значение $V = V_* = V^*$ называется бъемом области $G$
    \end{defenition}

    \begin{defenition}
        Говорят, что множество точек в $R^3$ имеет объем нуль, если все точки этого множества можно заключить в многогранник сколь угодно малого объема.
    \end{defenition}

    \section{Задача о вычислении массы тела}

    Пусть 

    \begin{enumerate}
        \item Тело $T$ занимает область $G \subset R^3$
        \item $f(x,y,z) \geq 0$ - значение плотности материала этого тела в точке $(x,y,z)$
    \end{enumerate}

    Требуется: Найти массу $m(T)$ тела $T$

    \begin{enumerate}
        \item Разобьем область $G$ на части:

            $$
            G = U_{i=1}^{n} G_i,\ \text{int }G_i \cap \text{int } G_j = 0, \text{ при } i \ne j
            $$

        \item В пределах кажддой из подобластей выберем отмеченную точку $M_i \in G_i,\ i = \overline{1;n}$

        \item Считая, что размеры $G_i$ малы:

            $$
            \Delta m_i = m(G_i) \approx f(M_i) \Delta V_i, \text{ где } \Delta V_i = V(G_i)
            $$

            масса тела, занимающего подобласть $G_i$

        \item Масса тела $T$ тогда:

            $$
            m(T) = \sum_{i=1}^n \Delta m_i \approx \sum_{i=1}^n f(M_i) \Delta V_i
            $$

        \item Эта формула тем точнее, чем меньше размеры $G_i$, поэтому естественно перейти к пределу:

            $$
            m(T) = \underset{ \underset{i}{\text{max}}\text{ diam} \to 0 }{\lim} \sum_{i=1}^n f(M_i) \Delta V_i
            $$
    \end{enumerate}

    \section{Определение тройного интеграла}

    Пусть 

    \begin{enumerate}
        \item $G \subseteq R^3$ - тело
        \item $f:G \to R$ - функция
    \end{enumerate}

    Разоьем область $G$ на части так, как это было сделано в задаче о вычислении массы тела

    Обозначим: $R = \{G_1,...,G_n\}$ - разбиение тела $G$

    \begin{defenition}
        Диаметром разбиения $R$ тела $G$ называется число

        $$
        d(R) = \underset{i = \overline{1;n}}{\text{max diam }} G_i
        $$
    \end{defenition}

    \begin{defenition}
        Тройным интегралом по функции $f(x,y,z)$ по области $G$ называется число 

        $$
        \iiint_G f(x,y,z) dxdydz = \lim_{d(R) \to 0} \sum_{i=1}^n f(M_i) \Delta V_i
        $$

        , где $M_i, \Delta V_i$ имееют тот же смысл, что и в задаче о вычислении массы тела
    \end{defenition}

    \begin{note}
        Если указанный в определении тройного интеграла предел $\exists$ и конечен, то функция $f$ называется интегрируемой в области 
    \end{note}

    \subsection{Свойства тройного интеграла}

    Эти свойства полностью аналогичны свойствам 1 - 9 двойного интеграла; при их записи нужно вместо $f(x,y) \mapsto f(x,y,z)$, $\iint_D f(x,y) dxdy \mapsto \iiint_G f(x,y,z) dxdydz$ $D \mapsto G$.

    \section{Вычисление тройного интеграла}

    Основная идея - сведение к повторному интегралу

    \begin{defenition}
        Область $G \subseteq R^3$ называется z-правильной, если любая прямая, параллельная $Oz$, пересекает границу $G$ не более чем в двух точках или содержит участок границы целиком.
    \end{defenition}

    z-правильная область $G$ можно задать в виде:

    $$
    G = \{ (x,y,z) : (x,y) \in D_{xy}, z_1(x,y) \le z \le Z_2(x,y) \}
    $$

    \begin{theorem}
        Пусть

        \begin{enumerate}
            \item $\exists \iiint_G f(x,y,z) dxdydz$
            \item $G$ является z-прав и задается (*)
            \item Для каждой фиксированной точки $(x,y) \in D_{xy}$
        \end{enumerate}

        $$
        \exists \int_{z_1(x,y)}^{z_2(x,y)} f(x,y,z) dz = F(x,y)
        $$

        Тогда 

        \begin{enumerate}
            \item $\exists$ повторный интеграл

                $$
                I_\text{повт} = \iint_{D_{xy}} F(x,y) dxdy = \iint_{D_{xy}} dxdy \int_{z_1(x,y)}^{z_2(x,y)} f(x,y,z) dz
                $$

            \item и $I_\text{повт} = I$
        \end{enumerate}
    \end{theorem}

    \begin{note}
        Если в условии * сформулированной теоремы область $D_{xy}$ является y-правильной и задается в виде:

        $$
        D_{xy} = \{ (x,y): a \le x \le b, \varphi_1(x) \le y \le \varphi_2(x) \}
        $$

        то

        $$
        \iiint_G f(x,y,z) dxdyd = \int_a^bdx \int_{\varphi_1(x)}^{\varphi_2(x)} dy \int_{z_1(x,y)}^{z_2(x,y)} f(x,y,z) dz
        $$
    \end{note}

    \section{Замена переменных в тройном итеграле}

    \begin{theorem}
        Пусть

        \begin{enumerate}
            \item $G_{xyz} = \varphi(G_{uv\omega})$
            \item $\varphi: G_{uv\omega} \to G_{xyz}$

                $$
                \varphi : 
                \begin{cases}
                    x = x(u, v, \omega) \\
                    y = y(u, v, \omega) \\
                    z = z(u, v, \omega) \\
                \end{cases}
                $$

            \item Отображение $\varphi$ биективно
            \item $\varphi$ непрерывно и непрырывно дифференцируемо в $G_{uv\omega}$
            \item 
                $$
                J_\varphi(u,v,\omega) = \left|
                \begin{matrix}
                    x'_u & x'_v & x'_\omega \\
                    y'_u & y'_v & y'_\omega \\
                    z'_u & z'_v & z'_\omega \\
                \end{matrix}
                \right| \ne 0
                $$
            \item $f(x,y,z)$ интегрируема в $G_{xyz}$
        \end{enumerate}

        Тогда

        $$
        \iiint_{G_{xyz}} f(x,y,z)dxdydz = \iiint_{G_{uv\omega}} f(x(u,v,\omega), y(u,v,\omega), z(u,v,\omega)) |J_\varphi(u,v,\omega)| dudvd\omega
        $$
    \end{theorem}

    \subsection{Связь цилиндрической и декартовой СК}

    $$
    \begin{cases}
        x = \rho \cos(\varphi) \\
        y = \rho \sin(\varphi) \\
        z = z \\
    \end{cases}
    $$

    $$
    J_\text{цил} = \left|
    \begin{matrix}
        \cos(\varphi) & -\rho \sin(\varphi) & 0 \\
        \sin(\varphi) & \rho \cos(\varphi) & 0 \\
        0 & 0 & 1
    \end{matrix}
    \right| = \rho
    $$

    \subsection{Связь сферической и декартовой СК}

    $$
    \begin{cases}
        x = r \cos(\Theta) \cos(\varphi) \\
        y = r \cos(\Theta) \sin(\varphi) \\
        z = r \sin(\Theta) \\
    \end{cases}
    $$

    $$
    |J_\text{сф}| = \dots = r^2 \cos(\Theta)
    $$

    \chapter{Определения вероятности}

    \section{Случайный эксперимент}

    \begin{defenition}
        Случайным называется эксперимент, результат которого невозможно предсказать.
    \end{defenition}

    \begin{enumerate}
        \item Подброс монетки

            $$
            \Omega = \{ \text{Г}, \text{Р}\}
            $$

            $$
            |\Omega| = 2
            $$

        \item Бросают игральную кость

            $$
            \Omega = \{ 1, 2, 3, 4, 5, 6 \}
            $$

            $$
            |\Omega| = 6
            $$

        \item Бросают монету до первого появления герба

            $$
            \Omega = \{ 1, 2, 3, \dots \}
            $$

            $$
            |\Omega| = \aleph_0
            $$

            Омега является счетным множеством, т.е. в нем столько же элементов, сколько существует натуральных чисел.

        \item Производят стрельбу по плоской мишени размеры которой 1м x 1м (координаты - точки попадания)

            $$
            \Omega = \{ (x,y) : |x| \le \frac{1}{2}; |y| \le \frac{1}{2} \}
            $$

            $$
            |\Omega| = c
            $$

            Омега имеет можность континуума
    \end{enumerate}

    \begin{defenition}
        Множество $\Omega$ всех исходов данного случайного эксперимента называется пространством элементов исхода
    \end{defenition}

    \begin{note}
        При рассматривании пространства элементов исходов предполагается, что
        \begin{enumerate}
            \item Каждый элемент исхода неделим, т.е. не может быть "разложен" на более мелкие исходы
            \item В результате случайного эксперимента всегда происходит ровно один элемент исхода из $\Omega$
        \end{enumerate}
    \end{note}

    \begin{defenition}
        (Нестрогое) Событием называется (любое) подмножество множества $\Omega$
    \end{defenition}

    \begin{defenition}
        Говорят, что в результате случайного эксперимента происходит событие $A$, если в результате этого эксперимента произошел один из входящих в $A$ элементов исхода.
    \end{defenition}

    Бросают игральную кость

    $$
    \Omega = \{ 1,2,3,4,5,6 \}
    $$

    $$
    A = \{2,4,6\}
    $$

    Если выпало 2 очко, то наступило $A$

    \begin{defenition}
        Событие $A$ называется следствием события $B$, если наступление события $B$ влечет наступление события $A$, т.е. $B \subseteq A$
    \end{defenition}

    \begin{note}
        Любое множество $\Omega$ содержит в качестве подмножеств $\emptyset$ и $\Omega$ соответствующие события называются невозможным ($\emptyset$) и достоверным ($\Omega$). Оба этих события называют несобственными. Все остальные события называют собственными.
    \end{note}

    В урне находится 2 красных и 3 синих шара. Из урны извлекают 1 шар

    $$
    A = \{ \text{извлеченный шар - зеленый} \} = \emptyset
    $$

    $$
    B = \{ \text{извлеченный шар - красный или синий} \} = \Omega
    $$

    \section{Операции над событиями}

    События - множества (подмножества множества $\Omega$) $\Rightarrow$ $\cup, \cap, \overline{a}, \backslash, \Delta$ 

    \begin{defenition}
        Суммой событий $A,B \subseteq \Omega$ называют событие

        $$
        A + B = A \cup B
        $$
    \end{defenition}

    \begin{defenition}
        Произведением событий $A,B \subseteq \Omega$ называют событие

        $$
        A \cdot B = A \cap B
        $$
    \end{defenition}

    \begin{defenition}
        $A \backslash B$ называется разностью событий $A$ и $B$
    \end{defenition}

    \begin{defenition}
        $\overline{A}$ называется событием, противоположным $A$
    \end{defenition}

    \subsection{Свойства операций над событиями}

    Смотреть теоретико-множеств. тождества (осно.)

    \begin{defenition}
        События $A,B \in \Omega$ называются несовместными, если $AB = \emptyset$. В противоположном случае события $A$ и $B$ называются совместными.
    \end{defenition}

    \begin{defenition}
        События $A_1, \dots, A_n, \dots$ называются попарнонесовместимимы, если $A_i A_j = \emptyset, i \ne j$ - несовместимыми в совокупности $A_1 \cdot \dots \cdot A_n = \emptyset$
    \end{defenition}

    \section{Классическое определение вероятности}

    Пусть

    \begin{enumerate}
        \item $|\Omega| = N < \infty$
        \item по условиям сложности эксперимента нет оснований предпочесть тот или иной исход остальных (в этом случае говорят, что все элементы исхода равновозможны)
        \item $A \subseteq \Omega,\ |A| = N_A$
    \end{enumerate}

    \begin{defenition}
        Вероятностью осуществления события $A$ называют число

        $$
        P(A) = \frac{N_A}{N}
        $$
    \end{defenition}

    2 раза бросают игральную кость

    $$
    A = \{ \text{сумма выпавших очков} \}
    $$

    $$
    P(A) = ?
    $$

    Решение:

    Исход: $(x_1, x_2)$, где $x_i$ - количество выпавших при $i$-м броске

    $$
    \Omega = \{ (1,1), (1,2), \dots, (6,6) \}
    $$

    $$
    |\Omega| = 36 = N
    $$
    
    $$
    A = \{ (5,6), (6,5), (6,6) \}
    $$

    $$
    N_A = |A| = 3 \\
    $$

    $$
    P(A) = \frac{N_A}{N} = \frac{3}{36} = \frac{1}{12}
    $$

\end{document}
