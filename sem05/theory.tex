\documentclass[a4paper, 14pt]{report}

\usepackage{cmap}
\usepackage[T2A]{fontenc}
\usepackage[utf8]{inputenc}
\usepackage[english,russian]{babel}

\usepackage{amsmath}
\usepackage{MnSymbol}
\usepackage{wasysym}

\author{Власов Павел Александрович}
\title{Теория вероятностей}
\date{2019}

\begin{document}
	\maketitle
	
	\tableofcontents
	\clearpage
	
	% ==================================================================
	% ==================================================================
	%					  ГЛАВА 1: ДВОЙНОЙ ИНТЕГРАЛ
	% ==================================================================
	% ==================================================================
	\chapter{Двойной интеграл}
	
	% ==================================================================
	%				  ЧАСТЬ 1: ПЛОЩАДЬ ПЛОСКОЙ ФИГУРЫ
	% ==================================================================
	\section{Площадь плоской фигуры}
	
	Пусть $D$ - фигура на плоскости.
	
	Как ввести понятие площади фигуры $D$?
	
	Если $D$ является треугольником (или прямоугольником) понятие площади очевидно.
	
	Если $D$ является многоугольником, то ее можно разбить на треугольники, а площадь области $D$ определить как сумму состовляющих ее треугольников.
	
	Что делать если $D$ - произвольная фигура
	
	\begin{enumerate}
	
		\item[а)] Рассмотрим множество многоугольников $M$, каждый из которых целиком содержится в $D$.
		
		Обозначим $S_* = \text{sup}\ S(m)$, где $m$ - многоугольники, $S(m)$ - площадь многоугольника
		
		\item[б)] Рассмотрим множество многоугольников $M$, каждый из которых содержит в себе $D$.
		
		Обозначим $S^* = \text{suf}\ S(M)$
	
	\end{enumerate}
	
	\underline{Определение} Область $D$ на плоскости называется \textbf{квадрируемой}, если $\exists$ конечные значения $S_*$, $S^*$ причем $S_* = S^*$. При этом число $S = S_* = S^*$ называется \textbf{площадью области $D$}
	
	\hfill
	
	\underline{Определение} Говорят, что множество $D$ точек плоскости имеет площадь \textbf{нуль}, если $D$ можно целиком заключить в многоугольник, сколь угодной площади, т.е. $\forall \varepsilon > 0\ \exists$ многоугольник $M$ площади $\varepsilon$ такой, что $D \le M$
	
	\hfill
	
	\textbf{Пример:}
	\begin{enumerate}
		\item[1)] $D = \{ A \}$, $A$ - точка
		\item[2)] $D = \{ AB \}$, $AB$ - отрезок
		\item[3)] Спрямленная (с конечной длиной) кривая
	\end{enumerate}
	
	\underline{Теорема} Пусть $D$ - замкнутая плоская область. Тогда $D$ - квадрируемая граница области $\Delta$. $\Leftrightarrow$ имеет площадь 0. $\blacksquare$
	
	\hfill
	
	\underline{Теорема} Пусть $\alpha$ - плоская спрямленная кривая. Тогда $\alpha$ - имеет площадь нуль. $\blacksquare$
	
	\hfill
	
	\underline{Следствие} Пусть 1) $D$ - область на плоскости, 2) $D$ ограничена конечным числом cпрямленных кривых. Тогда $D$ - квадрируема.
	
	\hfill
	
	\underline{Замечание:} В дальнейшем мы будем рассматривать только квадрируемые области
	
	% ==================================================================
	%	  ЧАСТЬ 2: ЗАДАЧИ, ПРИВОДЯЩИЕ К ПОНЯТИЮ ДВОЙНОГО ИНТЕГРАЛА
	% ==================================================================
	\section{Задачи, приводящие к понятию двойного интеграла}
	
	\begin{enumerate}
		\item[I.] \textbf{Задача об объеме цилиндрического тела}
		
		Пусть $D$ - область плоскости $Oxy$
		
		$f$: $D \to R$ - функция определенная на множестве $D$
		
		$$f(x, y) \geq 0\ \ \ (x, y) \in D$$
		
		Рассмотрим тело $T$, которое ограничено плоскостью $Oxy$, графиком функции $z = f(x,y)$ и цилиндрической поверхностью, направляющая которой совпадает с гранью $D$, а образующие параллельны $Oz$
		
		 \begin{enumerate}
		 	\item[1)] Разобьем область $D$ на пересекающиеся части
			
			$$ D = \overset{n}{\underset{i = 1}{U}} D_i $$
			
			$$ \text{int } D_i \cap \text{int } D_j = \emptyset, \text{ при } i \ne j\ (*)$$
			
			$$ \text{int } D_j - \text{множество внутренних точек области } D_i $$
			
			Условие (*) означает, что различные элементы разбиения не имеют общих внутренних точек
			
			\item[2)] Выберем точку $M_i \in D_i$ $i = \overline{1;n}$
			
			\item[3)] Считая, что размеры подобласти $D_i$ малы, примем $\Delta V_i \approx f(M_i) \Delta S_i$, где $\Delta S_i = S(D_i)$, $\Delta V_i$ - объем той части тела $T$, которая рассматривается под $D_i$
		 \end{enumerate}
		 
		\textbf{Тогда объем тела $T$:}
		
		$$ V = \sum_{i=1}^n \Delta V_i \approx \sum_{i=1}^n f(M_i) \Delta S_i $$
		
		Эта формула тем точнее, чем меньше размеры $D_i$, поэтому естественно перейти к пределу
		
		$$ V = \lim_{\underset{i = \overline{1,n}}{\text{max diam}}(D_i) \to 0} \sum_{i=1}^n f(M_i) \Delta S_i,$$
		
		$$\text{diam} (D) = \underset{M,N \in D}{\text{sup}} || \overline{MN} || - \text{диаметр множества } D$$
		
		\item[II.] \textbf{Задача о вычислении массы пластины}
		
		Пусть: 
		
		\begin{enumerate}
			\item[1)] Пластина занимает область $D$ на плоскости
			\item[2)] $T(x,y) \geq 0$ - плоскость поверхности материала пластины в точке $M(x,y)$
		\end{enumerate}
		
		Нужно найти массу $m$ этой частички
	\end{enumerate}
	
	\begin{enumerate}
		\item[1)] Разобьем область $D$ на непересекающей части
		
		$$ D = \overset{n}{\underset{i=1}{U}} D_i $$
		
		$$ \text{int } D_i \cap \text{int } D_j = \emptyset, i \ne j $$
		
		\item[2)] В пределах $D_i$ выберем точку $M_i, i = \overline{1,n}$
		\item[3)] Считая, что размеры $D_i$ малы, можно принять, что в пределах каждой из оластей $D_i$ плотность пластины меняется незначительно, поэтому во всех точках области $D_i$ плотность $\approx f(M_i)$ 
	
		Тогда масса части $D_i:\ \Delta m_i \approx f(M_i) \Delta S_i$, где $\Delta S_i = S(D_i),\ i = \overline{1, m}$
	
		\item[4)] \textbf{Тогда масса всей пластины}
		
		$$ m = \sum_{i=1}^n \Delta m_i \approx \sum_{i=1}^n f(M_i) \Delta S_i $$
	
	\end{enumerate}
	
	Полученная формула тем точнее, чем меньше размеры $D_i$, поэтому собственно
	
	$$ m = \lim_{\underset{i = \overline{1,n}}{\text{max diam}} (D_i) \to 0} \sum_{i=1}^n f(M_i) \Delta S_i $$
	
	% ==================================================================
	%	  ЧАСТЬ 3: ОПРЕДЕЛЕНИЕ СВОЙСТВА ДВОЙНОГО ИНТЕГРАЛА
	% ==================================================================
	\section{Определение свойства двойного интеграла}
	
	Пусть $D$ - квадратичная замкнутая плоская область
	
	\hfill
	
	\underline{Определение} \textbf{Разбиение области $D$} называется множество $R = \{ D_1, ... D_n\}$, где 
	
	\begin{enumerate}
		\item[1)] $D = \underset{i=1}{\overset{n}{U}} D_i$
		\item[2)] $\text{int } D_i \cap \text{int } D_j = \emptyset$, при $i \ne j$
		\item[3)] $D_i$ - квадрируема, $i = \overline{1,n}$
	\end{enumerate}
	
	\underline{Определение} \textbf{Диаметром разбиения} $R = \{ D_1, ... D_n \}$ называется число 
	
	$$d(R) = \underset{i = \overline{1,n}}{\text{max diam}} (D_i)$$
	
	Пусть $D$ - квадратичная замкнутая область на плоскости $Oxy$, $f: D \to R$ ($f$ является функцией двух переменных, т.к. $D$ - область на плоскости)
	
	\hfill
	
	\underline{Определение} \textbf{Двойным интегралом функции $f$ по области $D$} называется число 
	
	$$ \iint_D f(x,y) dx dy = \lim_{d(R) \to 0} \sum f(n_i) \Delta S_i, \text{ где}$$
	
	$$ R = \{D_1,...D_n\} \text{ - разбиение области } D $$
	$$ M_i \in D_i,\ i = \overline{1,n} \text{ - отмеченные точки} $$
	$$ \Delta S_i = S(D_i) $$
	
	\underline{Замечание} В определении подразумевается, что указанный предел существует, конечен и не зависит от разбиения $R$ области $D$ и способа выбора отмеченных точек
	
	\hfill
	
	\underline{Определение} Функции $f$, для которых существует $\iint_\Delta f dx dy$, называются \textbf{интегрируемыми в $D$}
	
	\hfill
	
	\textbf{Свойства двойного интеграла:}
	
	\begin{enumerate}
		\item[1)] $\iint_D 1 dxdy = S(D)$
		\item[2)] Линейность 
		
		Если $f,g$ - интегрируемы в $D$ функции, то 
		
		\begin{enumerate}
			\item[а)] $f \pm g$ интегрируема в $D$, $\iint_D (f \pm g) dxdy = \iint_D fdxdy \pm \iint_D g dxdy$
			\item[б)] $c \cdot f, c = \text{const}$ - интегрируема, $\iint_D c \cdot f dxdy = c \iint_D f dxdy$
		\end{enumerate}
		
		\item[3)] Аддитивность
		
		Пусть
		
		\begin{enumerate}
			\item[1.] $D_1,D_2$ - плоские квадратичные области
			\item[2.] $f$ интегрируема в $D_1$ и $D_2$
			\item[3.] $\text{int } D_1 \cap \text{int } D_2 = \emptyset$
		\end{enumerate}
		
		Тогда $f$ интегрируема в $D= D_1 \cup D_2$
		
		$$ \iint_D f dxdy = \iint_{D_1} f dxdy + \iint_{D_2} f dxdy $$
		
		\item[4)] О сохранении интегралом знака функции
		
		Пусть
		
		\begin{enumerate}
			\item[1.] $F(x,y) \geq 0$ в $D$
			\item[2.] $f$ - интегрируема в $D$
		\end{enumerate}
		
		тогда 
		
		$$ \iint_D F(x,y) dxdy \geq 0 $$
		
		\item[5)] Пусть
		
			\begin{enumerate}
			\item[1.] $ f(x,y) \geq g(x,y) $
			\item[2.] $ f,g $ - интегрируемы в $D$
		\end{enumerate}
		
		тогда
		
		$$ \iint_D f dxdy \geq \iint_D g dxdy $$
		
		\item[6)] Теорема об оценке модуля двойного интеграла
		
		Пусть $f$ интегрируема в $D$, тогда $|f|$ - интегрируем в $D$
		
		$$ |\iint_D f dxdy| \le \iint_D |f| dxdy $$
		
		\item[7)] Теорема об оценке двойного интеграла (обобщенная теорема)
		
		Пусть 
		
		\begin{enumerate}
			\item[1.] $f,g$ - интегрируемы в $D$
			\item[2.] $m \le f(x,y) \le M$
			\item[3.] $g(x,y) \geq 0$
		\end{enumerate}
		
		тогда
		
		$$ m \iint_D g(x,y) dxdy \le \iint_D f(x,y) dxdy \le M \iint_D g(x,y) dxdy $$
		
		\underline{Следствие} Если $g(x,y) \equiv 1$ в $D$, то получаем "просто" теорема об оценке двойного интеграла
		
		$$ m \cdot S \le \iint_D f(x,y) dxdy \le M \cdot S, \text{ где} S = S(D)$$
		
		
		\item[8)] Теорема о среднем значении
		
		\underline{Определение} \textbf{Средним значением функции $f$ в плоскости $D$} называется
		
		$$ <f> = {1 \over S(D)} \iint_D f(x,y) dxdy $$
		
		\underline{Свойство} Пусть 
		
		\begin{enumerate}
			\item[1.] $D$ - линейно связная замкнутая область (т.е. граница $D$ является связным множеством)
			\item[2.] $f$ - непрерывна в $D$
 		\end{enumerate}
		
		Тогда существует $M_0 \in D$, такая что $f(M_0) = <f>$
		
		\item[9)] Обобщенная теорема о среднем значении
		
		Пусть
		
		\begin{enumerate}
			\item[1.] $f$ - непрерывна в $D$
			\item[2.] $g$ - интегрируема в $D$
			\item[3.] $g$ - знакопостоянна
			\item[4.] $D$ - линейно связанной множество (если $f$ - непрерывна в $D$, то $f$ - интегрируема в $D$)
		\end{enumerate}
		
		тогда существует $M_0 \in D$ такая, что 
		
		$$ \iint_d f(x,y)g(x,y) dxdy = f(M_0) \iint_d g(x,y) dxdy $$
		
	\end{enumerate}
	
	\underline{Замечание} Свойство (8) является частным случаем свойства (9) для $g(x,y) = 1$
	
\end{document}